\documentclass[a4paper, openany]{memoir}

\usepackage[utf8]{inputenc}
\usepackage[T1]{fontenc} 
\usepackage[english]{babel}

\usepackage{fancyhdr}
\usepackage{float}
\usepackage{bm}

\usepackage{amsmath}
\usepackage{amsthm}
\usepackage{amssymb}
\usepackage{enumitem}
\usepackage{multicol}
\usepackage[bookmarksopen=true,bookmarksopenlevel=2]{hyperref}
\usepackage{tikz}
\usepackage{indentfirst}

\pagestyle{fancy}
\fancyhf{}
\fancyhead[LE]{\leftmark}
\fancyhead[RO]{\rightmark}
\fancyhead[RE, LO]{Topics In Algebra}
\fancyfoot[LE, RO]{\thepage}
\fancyfoot[RE, LO]{Pete Gautam}

\renewcommand{\headrulewidth}{1.5pt}

\theoremstyle{definition}
\newtheorem{definition}{Definition}[section]
\newtheorem{example}[definition]{Example}

\theoremstyle{plain}
\newtheorem{theorem}[definition]{Theorem}
\newtheorem{lemma}[definition]{Lemma}
\newtheorem{proposition}[definition]{Proposition}
\newtheorem{corollary}[definition]{Corollary}

\chapterstyle{thatcher}

\begin{document}
    \chapter{Review of 3H Algebra}
    \section{Isomorphism Theorems}

    \begin{theorem}[First Isomorphism Theorem]
        Let $G$ and $H$ be groups, and let $\varphi: G \to H$ be a homomorphism. Then, $G/\ker \varphi \cong \operatorname{Im}(\varphi)$.
    \end{theorem}
    \begin{proof}
        Let $H = \ker \varphi$. Define the map $\psi: G/H \to \operatorname{Im}(\varphi)$ by $\psi(gH) = \varphi(g)$. Let $g_1H, g_2H \in G/H$. We know that $g_2^{-1}g_1 \in H$, and so $\varphi(g_1) = \varphi(g_2)$. So, $\psi$ is well-defined. Moreover, since $\varphi$ is a homomorphism, we find that $\psi$ is a homomorphism. Also, by construction, $\psi$ is surjective.

        Now, we claim that $\psi$ is injective. Let $g_1H, g_2H \in G/H$ such that $\psi(g_1H) = \varphi(g_2H)$. In that case, $\varphi(g_1) = \varphi(g_2)$. Hence, $g_2^{-1}g_1 \in H$, meaning that $g_1H = g_2H$. This implies that $\psi$ is injective. So, $\psi$ defines an isomorphism.
    \end{proof}

    \begin{theorem}[Second Isomorphism Theorem]
        Let $G$ be a group, and let $H, N \leq G$ with $N \vartriangleleft G$. Then, $HN \leq G$, $H \cap N \vartriangleleft H$, and
        \[H/(H \cap N) \cong HN/N.\]
    \end{theorem}
    \begin{proof}
        Define the map $\varphi: H \to H/N$ by $\varphi(h) = hN$. This is a homomorphism, with
        \[\ker \varphi = \{g \in H \mid \varphi(g) = N\} = \{g \in H \mid g \in N\} = H \cap N,\]
        and
        \[\operatorname{Im} \varphi = \{hN \mid h \in H\} = HN/N.\]
        Hence, 
        \[H/(H \cap N) \cong HN/N.\]
    \end{proof}

    \begin{theorem}[Correspondence Theorem for Subgroups]
        Let $G$ be a group, and let $N \vartriangleleft G$. Then, there exists a bijection $f: S \to X$, where $S$ is the set of subgroups of $G$ containing $N$, and $X$ is the set of subgroups of $G/N$.
    \end{theorem}
    \begin{proof}
        Let $q: G \to G/N$ be the quotient map. Define the map $f: S \to X$ by 
        \[f(H) = q(H) = \{hN \mid h \in H\} =: H/N.\]
        We show that $f$ is bijective. Let $L \leq G/N$. Then, set 
        \[K = q^{-1}(L) = \{g \in G \mid gN \in L\}\]
        is a subgroup of $G$. We have $N \in L$, so $N \leq K$. This implies that $K \in S$. Moreover,
        \[gN \in L \iff g \in K \iff gN \in K/N.\]
        So, $L = K/N$. This implies that $f$ is surjective. Also, for $H/N = K/N$, we have
        \[g \in H \iff gN \in H/N \iff gN \in K/N \iff g \in K.\]
        So, $H = K$. This implies that $f$ is injective as well. Hence, $f$ is a bijection.
    \end{proof}

    \begin{theorem}[Third Isomorphism Theorem]
        Let $G$ be a group, and let $H, K \vartriangleleft G$, wth $K \leq H$. Then,
        \[(G/K)/(H/K) \cong G/H.\]
    \end{theorem}
    \begin{proof}
        Define the map $\psi: G/K \to G/H$ by $\psi(gK) = gH$. For $g_1K, g_2K \in G/H$, if $g_1K = g_2K$, then $g_2^{-1}g_1 \in K \subseteq H$. So, $g_1H = g_2H$, meaning that $\psi$ is well-defined. Moreover, the map $\psi$ is surjective by construction. The map $\psi$ is also a homomorphism by definition of quotients. Now,
        \[\ker \psi = \{gK \in G/K \mid gK = H\} = \{gK \in G/K \mid g \in H\} = H/K.\]
        So, the First Isomorphism Theorem tells us that
        \[(G/K)/(H/K) \cong G/H.\]
    \end{proof}

    \newpage

    \section{Intersection, Product and Join}
    \begin{proposition}
        Let $G$ be a group and $H, K \leq G$ with $H \vartriangleleft G$. Then, $HK \leq G$.
    \end{proposition}

    \begin{definition}
        Let $G$ be a group and $H, K \leq G$. Then, the \emph{join} of $H$ and $K$ is given by
        \[H \wedge K := \bigcap_{\substack{N \leq G \\ H, K \leq N}} N.\]
    \end{definition}

    \begin{proposition}
        Let $G$ be a group and $H, K \leq G$. Then, $HK = H \wedge K$ if and only if $HK \leq G$.
    \end{proposition}
    \begin{proof}
        If $HK = H \wedge K$, then $HK \leq G$. So, assume that $HK \leq G$. We have $H, K \leq HK$, so $H \wedge K \leq HK$ by definition. Now, let $hk \in HK$ and $N \leq G$ such that $H, K \leq N$. Then, $h, k \in N$, meaning that $hk \in N$. Hence, $hk \in H \wedge K$. So, $HK = H \wedge K$.
    \end{proof}

    \begin{proposition}
        Let $G$ be a group and $H, K \leq G$ be finite. Then,
        \[|HK| = \frac{|H| |K|}{|H \cap K|}.\]
    \end{proposition}
    \begin{proof}
        We know that 
        \[HK = \{hk \mid h \in H, k \in K\} = \bigcup_{h \in H} hK.\]
        So, the cardinality of $HK$ is the number of distinct left cosets $hK$ for $h \in H$. We know that for all $h_1, h_2 \in H$,
        \begin{align*}
            h_1K = h_2K &\iff h_2^{-1}h_1 = K \\
            &\iff h_2^{-1}h_1 = H \cap K \\
            &\iff h_1(H \cap K) = h_2(H \cap K).
        \end{align*}
        Hence, the number of distinct left cosets $hK$ for $h \in H$ is equal to the number of distinct left cosets $h(H \cap K)$ for $h \in H$. We note that $H \cap K \leq H$, with the number of left cosets given by
        \[[H : H \cap K] = \frac{|H|}{|H \cap K|}.\]
        Hence, there are $\frac{|H|}{|H \cap K|}$ distinct left cosets $hK$ for $h \in H$. Since each coset has $|K|$ elements, it follows that 
        \[|HK| = \frac{|H| |K|}{|H \cap K|}.\]
    \end{proof}
    \newpage
    
    \section{Composition Series}
    \begin{definition}
        Let $G$ be a group, and let $H_i \leq G$ for all $i \in \{1, \dots, n-1\}$. We say that 
        \[\{e\} = H_0 \leq H_1 \leq \dots \leq H_{n-1} \leq H_n = G\]
        is a \emph{group series} if $H_i \leq H_{i+1}$ for all $i \in \{0, \dots, n-1\}$. The group series
        \[\{e\} = H_0 \leq H_1 \leq \dots \leq H_{n-1} \leq H_n = G\]
        is a \emph{normal} series if $H_i \vartriangleleft G$ for all $i \in \{0, \dots, n-1\}$. Also, the group series
        \[\{e\} = H_0 \leq H_1 \leq \dots \leq H_{n-1} \leq H_n = G\]
        is \emph{subnormal} if $H_i \vartriangleleft H_{i+1}$ for all $i \in \{0, \dots, n-1\}$.
    \end{definition}

    \begin{definition}
        Let $G$ be a group, and let 
        \[\{e\} = H_0 \leq H_1 \leq \dots \leq H_{n-1} \leq H_n = G\]
        be a subnormal series. We say that the group series is a \emph{composition series} if for all $n \in \{0, \dots, n-1\}$, $H_i/H_{i+1}$ is simple. If 
        \[\{e\} = H_0 \leq H_1 \leq \dots \leq H_{n-1} \leq H_n = G\]
        is a normal series such that for all $n \in \{0, \dots, n-1\}$, $H_i/H_{i+1}$ is simple, then the group series is a \emph{principal series}.
    \end{definition}

    \begin{proposition}
        $\mathbb{Z}$ has no composition series.
    \end{proposition}
    \begin{proof}
        Let the following be a subnormal series for $\mathbb{Z}$:
        \[\{0\} = G_0 \vartriangleleft G_1 \vartriangleleft \dots G_n = \mathbb{Z}.\]
        We know that the subgroup $G_1 = m\mathbb{Z}$, for some $m \in \mathbb{Z}$. Then, the quotient $G_1/G_0 \cong m\mathbb{Z}$ is not simple. So, the subnormal series is not a composition series.
    \end{proof}

    \begin{lemma}[Zussenhaus Lemma]
        Let $G$ be a group, with normal subgroups $H$ and $K$, and let $H^* \vartriangleleft H$ and $K^* \vartriangleleft K$. Then,
        \begin{itemize}
            \item $(H \cap K)H^*$ and $(H \cap K)K^*$ are subgroups of $G$;
            \item $(H \cap K^*)H^* \vartriangleleft (H \cap K)H^*$ and $(H^* \cap K)K^* \vartriangleleft (H \cap K)K^*$;
            \item \[ (H \cap K)H^* / (H \cap K^*)H^* \cong (H \cap K)K^* / (H^* \cap K)K^*.\]
        \end{itemize}
    \end{lemma}
    \begin{definition}
        Let $G$ be a group, and consider the following subnormal series of $G$:
        \[\{e\} = H_0 \leq H_1 \leq \dots \leq H_n = G, \qquad \{e\} \leq K_0 \leq K_1 \leq \dots \leq K_m = G.\]
        We say that the subnormal series $(H)$ and $(K)$ are \emph{isomorphic} if there exists a bijection $\sigma: \{1, \dots, n\} \to \{1, \dots, m\}$ such that for all $1 \leq i \leq n-1$, $H_{i+1}/H_i \cong K_{\sigma(i)+1}/K_{\sigma(i)}$.
    \end{definition}

    \begin{theorem}[Schreier Refinement Theorem]
        Let $G$ be a group, and consider the following subnormal series of $G$:
        \[\{e\} = H_0 \leq H_1 \leq \dots \leq H_n = G, \qquad \{e\} \leq K_0 \leq K_1 \leq \dots \leq K_m = G.\]
        Then, there exist isomorphic subnormal series
        \[\{e\} = H_0' \leq H_1' \leq \dots \leq H_a' = G, \qquad \{e\} \leq K_0' \leq K_1' \leq \dots \leq K_a' = G\]
        such that $(H_i')$ refines $(H_i)$ and $(K_i')$ refines $(K_i)$.        
    \end{theorem}

    \begin{example}
        We illustrate the Schreier Theorem with an example. Consider the following subnormal series of $\mathbb{Z}$:
        \[\{0\} \leq 5\mathbb{Z} \leq \mathbb{Z}, \qquad \{0\} \leq 18\mathbb{Z} \leq 6\mathbb{Z} \leq \mathbb{Z}.\]
        Then, the left series is refined as follows:
        \[\{0\} \leq \{0\} + (5\mathbb{Z} \cap 18\mathbb{Z}) \leq \{0\} + (5\mathbb{Z} \cap 6\mathbb{Z}) \leq 5\mathbb{Z} \leq 5\mathbb{Z} + (\mathbb{Z} \cap 18\mathbb{Z}) \leq 5\mathbb{Z} + (\mathbb{Z} \cap 6\mathbb{Z}) \leq \mathbb{Z}.\]
        This simplifies to the following subnormal series:
        \[\{0\} \leq 90\mathbb{Z} \leq 30\mathbb{Z} \leq 5\mathbb{Z} \leq \mathbb{Z}.\]
        So, the composition factors are: $90\mathbb{Z}, \mathbb{Z}_3, \mathbb{Z}_6, \mathbb{Z}_5$. 
        
        Now, the right series is refined as follows:
        \[\{0\} \leq \{0\} + (18\mathbb{Z} \cap 5\mathbb{Z}) \leq 18\mathbb{Z} \leq 18\mathbb{Z} + (6\mathbb{Z} \cap 5\mathbb{Z}) \leq 6\mathbb{Z} \leq 6\mathbb{Z} + (\mathbb{Z} \cap 5\mathbb{Z}) \leq \mathbb{Z}.\]
        This simplifies to the following subnormal series:
        \[\{0\} \leq 90\mathbb{Z} \leq 18\mathbb{Z} \leq 6\mathbb{Z} \leq \mathbb{Z}.\]
        Here, the composition factors are: $90\mathbb{Z}, \mathbb{Z}_5, \mathbb{Z}_3, \mathbb{Z}_6$. So, the two subnormal series are isomorphic.
    \end{example}

    \begin{theorem}[Jordan-Holder Theorem]
        Let $G$ be a group, and consider the following composition series of $G$:
        \[\{e\} = H_0 \leq H_1 \leq \dots \leq H_n = G, \qquad \{e\} \leq K_0 \leq K_1 \leq \dots \leq K_m = G.\]
        Then, the composition series are isomorphic. 
    \end{theorem}
    \begin{proof}
        We prove by induction on the length of the shortest composition series. If the composition series has length 0, then $G$ is trivial. In that case, the statement is trivial- there is only one composition series of $G$. Now, assume that for a group $G$ with composition series of length $n-1$, the statement holds. Next, let $G$ have a composition series of shortest length $n$, which is given by
        \[\{e\} = H_0 \leq H_1 \leq \dots \leq H_n = G.\]
        Moreover, let $G$ have another composition series
        \[\{e\} \leq K_0 \leq K_1 \leq \dots \leq K_m = G,\]
        with $m \geq n$. Let $H = H_{n-1}$ and $K = K_{m-1}$. If $H = K$, then the induction hypothesis tells us that the group series
        \[\{e\} = H_0 \leq H_1 \leq \dots \leq H_{n-1} \qquad \text{and} \qquad \{e\} \leq K_0 \leq K_1 \leq \dots \leq K_{m-1}\]
        are isomorphic. Hence, the group series
        \[\{e\} = H_0 \leq H_1 \leq \dots \leq H_n = G \qquad \text{and} \qquad \{e\} \leq K_0 \leq K_1 \leq \dots \leq K_m = G\]
        are isomorphic.

        Now, assume that $H \neq K$. In that case, let $L = H \cap K$. Since $H$ and $K$ are normal in $G$, $L$ is normal in $G$, and so $L$ is normal in $H$ and $K$. Define now the group series
        \[\{e\} = L_0 \leq L_1 \leq \dots \leq L_{n-2} \leq L_{n-1} = L,\]
        where $L_i = L \cap H_i$ for $0 \leq i \leq n-1$. We claim that this is a composition series. We know that $L_i = L \cap H_i$ and $L_{i+1} = L \cap H_{i+1}$. Since $H_i \vartriangleleft H_{i+1}$, the second isomorphism theorem tells us that $L \cap H_i \vartriangleleft L$. Hence, $L \cap H_i \vartriangleleft L \cap H_{i+1}$, i.e. $L_i \vartriangleleft L_{i+1}$. Consider the quotient map from $H_{i+1}$ to $H_{i+1}/H_i$, restricted to $L_{i+1}$- $\varphi\colon L_{i+1} \to H_{i+1}/H_i$. We have 
        \[\ker \varphi = \{x \in L_{i+1} \mid x \in H_i\} = L_{i+1} \cap H_i = (L \cap H_{i+1}) \cap H_i = L \cap H_i = L_i.\]
        Hence, the first isomorphism theorem tells us that
        \[L_{i+1}/L_i \cong \operatorname{im} \varphi = L_{i+1}/H_i.\]
        We know that $L \vartriangleleft H$ and $H_{i+1} \leq H$. Hence, $L_{i+1} = L \cap H_{i+1} \vartriangleleft H_{i+1}$. So, $L_{i+1}/H_i \vartriangleleft H_{i+1}/H_i$. This implies that $L_{i+1}/H_i$ is simple (or trivial). Hence, $L_{i+1}/L_i$ is simple (or trivial)- this is a composition series.

        Now, since $H, K \vartriangleleft G$, $HK \vartriangleleft G$. This implies that $HK/K \vartriangleleft G/K$. The group $G/K$ is simple, and since $H \neq K$, we find that $HK/K = G/K$. By Correspondence Theorem, we find that $G = HK$. Hence, the second isomorphism theorem tells us that
        \begin{align*}
            H/(H \cap K) \cong HK/K &\iff H/L \cong G/K \\
            K/(H \cap K) \cong HK/K &\iff K/L \cong G/H.
        \end{align*}
        
        Next, consider the following two composition series:
        \[\{e\} = H_0 \leq H_1 \leq \dots \leq H_{n-1} = H, \quad \{e\} \leq L_0 \leq L_1 \leq \dots \leq L_{n-1} = L \leq H.\]
        The left composition series has length $n-1$. So, by the induction hypothesis, the left and the right composition series are isomorphic. In particular, the length of the right composition series (without duplicate) is $n-1$. We also have the following two composition series:
        \[\{e\} = K_0 \leq K_1 \leq \dots \leq K_{m-1} = K, \quad \{e\} \leq L_0 \leq L_1 \leq \dots \leq L_{n-1} = L \leq K.\]
        Since $H \neq K$, $L \neq K$. Hence, the right composition series still has length $n-1$. So, the induction hypothesis tells us that the two composition series are isomorphic. Finally, consider the following composition series:
        \[\{e\} \leq L_0 \leq L_1 \leq \dots \leq L \leq H \leq G, \quad \{e\} \leq L_0 \leq L_1 \leq \dots \leq L \leq K \leq G.\]
        We have established that these composition series are isomorphic to the original composition series- they are isomorphic up to the second last subgroup, and the last factor is equal. So, it suffices to show that these composition series are isomorphic. The only difference in these composition series is the final two factors- $\{H/L, G/H\}$ and $\{K/L, G/K\}$. We know that $H/L \cong G/K$ and $K/L \cong G/H$, so these factors are isomorphic. Hence, the result follows by induction.
    \end{proof}

    \begin{definition}
        Let $G$ be a group. We say that $G$ is \emph{solvable} if there exists a subnormal series
        \[\{e\} = H_0 \leq H_1 \leq \dots \leq H_{n-1} \leq H_n = G\]
        such that for all $0 \leq i < n$, $H_{i+1}/H_i$ is abelian.
    \end{definition}

    \begin{example}
        The group $S_5$ is not solvable.
    \end{example}
    \begin{proof}
        A composition series for $S_5$ is:
        \[\{e\} \leq A_5 \leq S_5.\]
        The quotients are $A_5$ and $\mathbb{Z}_2$. By Jordan-Holder, we know that any composition series of $S_5$ will have these quotients. Since $A_5$ is not abelian, we find that $S_5$ is not solvable.
    \end{proof}

    \begin{proposition}
        Let $G$ be a group and let $N \vartriangleleft G$. Then, $G$ is solvable if and only if $N$ and $G/N$ are solvable.
    \end{proposition}
    \begin{proof}
        First, assume that $N$ and $G/N$ are solvable. In that case, we can find a subnormal series
        \[\{e\} = N_0 \leq N_1 \leq \dots \leq N_{k-1} \leq N_k = N\]
        such that for all $0 \leq i < k$, $N_{i+1}/N_i$ is abelian. Moreover, there exists a subnormal series 
        \[\{N\} = G_0/N \leq G_1/N \leq \dots \leq G_l/N = G/N\]
        such that for all $0 \leq i < l$, $(G_{i+1}/N)/(G_i/N)$ is abelian. Now, consider the group series
        \[\{e\} = L_0 \leq L_1 = N_1 \leq \dots \leq L_k = N_k \leq L_{k+1} = G_1 \leq \dots \leq L_{k+l} = G.\]
        We claim that for all $0 \leq i < k+l$, $L_{i+1}/L_i$ is abelian. Clearly, this holds if $0 \leq i < k$. Now, we find that for $k < i < k+l$,
        \[L_{i+1}/L_i = (G_{i-k+1}/N)/(G_{i-k}/N) \cong G_{i-k+1}/G_{i-k}\]
        by the third isomorphism theorem. Finally, we know that $G_1/N$ is abelian since $G_0/N$ is trivial. Hence, it is a group series with abelian factors. This implies that $G$ is solvable.

        Now, assume that $G$ is solvable. So, we can find a subnormal series
        \[\{e\} = G_0 \leq G_1 \leq \dots \leq G_{k-1} \leq G_k = G\]
        such that for all $0 \leq i < k$, $G_{i+1}/G_i$ is abelian. Now, for each $0 \leq i \leq k$, define the group $N_i = G_i \cap N$. Consider the group series
        \[\{e\} = N_0 \leq N_1 \leq \dots \leq N_{k-1} \leq N_k = N.\]
        We know that for all $0 \leq i < n$, $G_i \vartriangleleft G_{i+1}$. So, the second isomorphism theorem tells us that $N_i = N \cap G_i \vartriangleleft N$. Hence, $N_i \vartriangleleft N \cap G_{i+1} = N_{i+1}$. Next, consider the quotient map $\varphi \colon N_{i+1} \to N_{i+1}/G_i$. We find that
        \[\ker \varphi = \{g \in N_{i+1} \mid gG_i = G_i\} = N_{i+1} \cap G_i = N_i.\]
        Hence, the first isomorphism theorem tells us that
        \[N_{i+1}/N_i \cong N_{i+1}/G_i \leq G_{i+1}/G_i.\]
        Since $G_{i+1}/G_i$ is abelian, it follows that $N_{i+1}/N_i$ is abelian. Hence, $N$ is solvable. Next, consider the group series
        \[\{N\} = G_0/N \leq G_1/N \leq \dots \leq G_{k-1}/N \leq G_k/N = G/N.\]
        By the third isomorphism theorem, we know that for all $0 \leq i < k$,
        \[(G_i/N)/(G_{i+1}/N) \cong G_i/G_{i+1}.\]
        Hence, $(G_i/N)/(G_{i+1}/N)$ is abelian. So, $G/N$ is also solvable.
    \end{proof}

    \begin{definition}
        Let $G$ be a group, and let $H = G/Z(G)$. Define the subgroup $Z_1(G) = q^{-1}(Z(H))$, where $q \colon G \to H$ is the quotient map. We define the \emph{ascending central series} of $G$ to be the following group series as follows:
        \begin{itemize}
            \item if $G = Z_1(G)$, then
            \[\{e\} \leq Z(G) \leq G.\]
            \item otherwise, we define
            \[\{e\} \leq Z(G) \leq Z_1(G) \leq Z_2(G) \leq \dots \leq G,\]
            where $Z_2(G)$ can be constructed the same way we constructed $Z_1(G)$- the series terminates when $Z_n(G) = G$, if possible.
        \end{itemize}
        If $G$ has a finite ascending central series, then $G$ is \emph{nilpotent}.
    \end{definition}

    \begin{example}
        Let $G = S_3$. Then, $Z(G)$ is trivial, meaning that $Z_1(G)$ is trivial too. So, it has the following ascending central series:
        \[\{()\} \leq \{()\}  \leq \dots \]
        This does not terminate, so $S_3$ is not nilpotent. Now, let $G = \mathbb{Z}_6$. Then, $Z(G) = G$, meaning that it has the following ascending central series:
        \[\{0\} \leq \mathbb{Z}_6.\]
        Finally, let $G = D_4$. Then, $Z(G) = \langle r^2 \rangle$, meaning that $G/Z(G)$ is abelian. Hence, it has the following ascending central series:
        \[\{e\} \leq \{e, r^2\} \leq D_4.\]
    \end{example}

    \begin{definition}
        Let $G$ be a group. We defined the \emph{derived series} for $G$ to be the following subnormal series:
        \[G = G_0 \geq G_1 \geq G_2 \geq \dots\]
        where $G_i = [G_{i-1}, G_{i-1}]$. We say that the derived series terminates if there exists an $n \in \mathbb{Z}_{\geq 1}$ such that $G_n$ is trivial.
    \end{definition}

    \begin{example}
        Let $G = S_3$. Then, $[G, G] = A_3$\sidefootnote{It is the smallest normal subgroup such that the quotient is abelian.}. Since $A_3$ is abelian, we have $[A_3, A_3] = 1$. Hence, the derived series for $G$ is:
        \[S_3 \geq A_3 \geq \{()\}.\]
        Next, if $G = A_5$, then $[G, G] = A_5$. So, the derived series is:
        \[A_5 \geq A_5 \geq \dots\]
        So, the series does not terminate.
    \end{example}

    \begin{proposition}
        Let $G$ be a group. Then, $G$ is solvable if and only if the derived series terminates.
    \end{proposition}
    \begin{proof}
        Assume that the derived series terminates. Hence,
        \[G = G_0 \geq G_1 \geq \dots \geq G_n = \{e\}\]
        is a subnormal series for $G$. We know that $G_i = [G_{i-1}, G_{i-1}]$, hence $G_i/G_{i+1}$ is abelian. So, $G$ is solvable.

        Now, assume that $G$ is solvable. In that case, there exists a subnormal series
        \[\{e\} = H_0 \leq H_1 \leq \dots \leq H_n = G\]
        such that $H_{i+1}/H_i$ is abelian. In particular, we have $[H_{i+1}, H_{i+1}] \leq H_i$. Since $[H_{i+1}, H_{i+1}] \vartriangleleft H_{i+1}$, we find that $[H_{i+1}, H_{i+1}] \vartriangleleft H_i$. So, we can refine the subnormal series to its derived series. This implies that the derived series terminates.
    \end{proof}

    \begin{definition}
        Let $G$ be a group. We say that $G$ is \emph{perfect} if its commutator subgroup $[G, G] = G$.
    \end{definition}
\end{document}