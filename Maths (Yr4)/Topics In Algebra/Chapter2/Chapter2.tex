\documentclass[a4paper, openany]{memoir}

\usepackage[utf8]{inputenc}
\usepackage[T1]{fontenc} 
\usepackage[english]{babel}

\usepackage{fancyhdr}
\usepackage{float}
\usepackage{bm}

\usepackage{amsmath}
\usepackage{amsthm}
\usepackage{amssymb}
\usepackage{enumitem}
\usepackage{multicol}
\usepackage[bookmarksopen=true,bookmarksopenlevel=2]{hyperref}
\usepackage{tikz}
\usepackage{indentfirst}

\pagestyle{fancy}
\fancyhf{}
\fancyhead[LE]{\leftmark}
\fancyhead[RO]{\rightmark}
\fancyhead[RE, LO]{Topics In Algebra}
\fancyfoot[LE, RO]{\thepage}
\fancyfoot[RE, LO]{Pete Gautam}

\renewcommand{\headrulewidth}{1.5pt}

\theoremstyle{definition}
\newtheorem{definition}{Definition}[section]

\theoremstyle{plain}
\newtheorem{theorem}[definition]{Theorem}
\newtheorem{lemma}[definition]{Lemma}
\newtheorem{proposition}[definition]{Proposition}
\newtheorem{corollary}[definition]{Corollary}
\newtheorem{example}[definition]{Example}


\chapterstyle{thatcher}
\setcounter{chapter}{1}

\begin{document}
    \chapter{Finite Groups}
    \section{Cauchy Theorem}

    \begin{lemma}
        Let $G$ be a $p$-group and $X$ be a finite set that $G$ acts on. Then, $|X^G| \equiv |X| \bmod{p}$.
    \end{lemma}
    \begin{proof}
        Let $|G| = p^n$. Since orbits form an equivalence relation, we know that
        \[|X| = |X^G| + \sum_{i=1}^n p^i n_i,\]
        where $n_i$ is the number of orbits of length $p^i$. Hence, $|X^G| \equiv |X| \bmod{p}$.
    \end{proof}

    \begin{proposition}
        Let $G$ be a $p$-group. Then, $Z(G)$ is not trivial.
    \end{proposition}
    \begin{proof}
        Let $G$ act on itself via conjugation. Then, the fixed points are given by
        \[\{x \in G \mid gxg^{-1} = x \forall g \in G\} = \{x \in G \mid gx = xg \forall g \in G\} = Z(G).\]
        So, the lemma above tells us that $|Z(G)| \equiv |G| \bmod{p}$. Since $G$ is a $p$-group, we find that $|Z(G)| \equiv 0 \bmod{p}$. We have $e \in Z(G)$, so we find that $|Z(G)| \geq p$. Hence, $Z(G)$ is not trivial.
    \end{proof}

    \begin{corollary}
        Let $G$ be a group of order $p^2$, for some prime $p$. Then, $G$ is abelian.
    \end{corollary}
    \begin{proof}
        We know that $Z(G)$ is a subgroup of $G$. So, $|Z(G)| \in \{1, p, p^2\}$. By the proposition above, we find that $|Z(G)| \neq 1$. Now, assume that $|Z(G)| = p$. Then, let $g \in G$ with $g \not\in Z(G)$. Consider the centraliser subgroup $C_G(g)$. We have $Z(G) \leq C_G(g)$, and $g \in C_G(g)$. Hence $|C_G(g)| \geq p + 1$. By Lagrange's Theorem, we must have that $C_G(g) = G$. That is, for all $x \in G$, $gx = xg$, meaning that $g \in Z(G)$. This is a contradiction. So, we must have $|Z(G)| = p^2$. Hence, $G$ is abelian.
    \end{proof}

    \begin{theorem}[Cauchy Theorem]
        Let $G$ be a finite group of order dividing some prime $p$. Then, there exists a $g \in G$ with $|g| = p$.
    \end{theorem}
    \begin{proof}
        Define the set
        \[X = \{(g_1, \dots, g_p) \mid g_1 \dots g_p = e\}.\]
        By construction, $|X| = |G|^{p-1}$- we can freely choose $g_1, \dots, g_{p-1}$ and set $g_p = (g_1 \dots g_{p-1})^{-1}$. Since $G$ has order dividing $p$, $|X| \equiv 0 \bmod{p}$. We can define the action from the group $\mathbb{Z}/p\mathbb{Z}$ on the set $X$ by reorder, i.e.
        \[(1 + p\mathbb{Z}) \cdot (g_1, g_2, \dots, g_p) = (g_2, \dots, g_p, g_1),\]
        extended as a homomorphism. This is well-defined since
        \[g_1g_2 \dots g_p = e \iff g_2 \dots g_p = g_1^{-1} \iff g_2 \dots g_p g_1 = e.\]
        Now, since $\mathbb{Z}/p\mathbb{Z}$ is a $p$-group, we find that $|X^G| \equiv 0 \bmod{p}$. We have
        \begin{align*}
            X^G &= \{(g_1, \dots, g_p) \in X \mid  n \cdot (g_1, \dots, g_p) = (g_2, \dots, g_1) \ \forall n \in \mathbb{Z}/p\mathbb{Z} \} \\
            &= \{(g, \dots, g) \mid g \in G, g^p = e\}.
        \end{align*}
        We know that the element $(e, \dots, e) \in X^G$. So, $|X^G| \geq 1$. Since $|X^G| \equiv 0 \bmod{p}$, we find that $|X^G| \geq p > 1$. In particular, there exists a $g \in G$ with $g \neq e$ such that $g^p = e$. Hence, $|g| = p$.
    \end{proof}

    \newpage

    \section{Sylow Theorems}
    \begin{lemma}
        Let $G$ be a group of order $p^nm$, for a prime $p$ with $(p, m) = 1$, and let $H \leq G$ be a $p$-subgroup. Then, $[G:H] \equiv [N_G(H):H] \bmod{p}$. In particular, if $H$ has order $p^i$ for $i < n$, then $N_G(H) \neq H$.
    \end{lemma}
    \begin{proof}
        Let $X = G/H$ be the set of left cosets of $H$ in $G$. Then, $H$ acts on $X$ by left multiplication, i.e. $h \cdot gH = hgH$. We have
        \begin{align*}
            X^H &= \{gH \in G/H \mid hgH = gH \ \forall h \in H\} \\
            &= \{gH \in G/H \mid g^{-1}hg \in H \ \forall h \in H\} \\
            &= \{gH \in G/H \mid gHg^{-1} = H\}= N_G(H)/H.
        \end{align*}
        Hence,
        \[[G:H] = |X| \equiv |X^H| = [N_G(H):H] \bmod{p}.\]
    \end{proof}

    \begin{theorem}[Sylow I]
        Let $G$ be a group of order $p^nm$, for a prime $p$ with $(p, m) = 1$. Then, for all $1 \leq i \leq n$, $G$ has a subgroup of order $p^i$. Moreover, for all $1 \leq i < n$ and a subgroup $H_i$ of order $p^i$, there exists a subgroup $K_i$ of order $p^{i+1}$ such that $H_i \vartriangleleft K_i$.
    \end{theorem}
    \begin{proof}
        We show this by induction. By Cauchy's Theorem, we know that there exists a subgroup of order $p$, so the statement holds for $i = 1$. Now, assume the statement holds for some $1 \leq i < n$. Hence, there exists a subgroup $H_i \leq G$ of order $p^i$. By the result above, we know that $N_G(H_i) \neq H_i$. Therefore, the quotient $N_G(H_i)/H_i$ is not trivial. Moreover, $[G:H_i] \equiv 0 \bmod{p}$ since $i \neq n$. Hence, $[N_G(H_i):H_i] \equiv 0 \bmod{p}$. By Cauchy's Theorem, there exists a $gH_i \in N_G(H_i)/H_i$ such that $|gH_i| = p$. We know that $\langle gH \rangle = H_{i+1}/H_i$ by correspondence theorem, for some subgroup $H_i \leq H_{i+1} \leq N_G(H_i)$. Since $|H_{i+1}/H_i| = p$, we find that $|H_{i+1}| = p^{i+1}$. So, there exists a subgroup $H_{i+1}$ of order $p^{i+1}$. Moreover, since $H_{i+1} \leq N_G(H_i)$, we must have $H_i \vartriangleleft H_{i+1}$. So, the result follows from induction.
    \end{proof}

    \begin{definition}
        Let $G$ be a group of order $p^nm$, for a prime $p$ with $(p, m) = 1$, and let $H \leq G$. We say that $H$ is a \emph{Sylow-$p$ subgroup} if $|H| = p^n$.
    \end{definition}

    \begin{theorem}[Sylow II]
        Let $G$ be a group of order $p^nm$, for a prime $p$ with $(p, m) = 1$. Then, all the Sylow-$p$ subgroups are conjugate.
    \end{theorem}
    \begin{proof}
        Let $H$ and $K$ be Sylow-$p$ subgroups. We show that $H$ and $K$ are conjugate. Let $X = G/H$ be the set of left cosets of $H$ in $G$. Then, $K$ acts on $X$ by left multiplication, i.e. $k \cdot gH = kgH$. We know that $|X^K| \equiv X \bmod{p}$. Since $|X^K| = m$, we find that $|X^K| \not\equiv 0 \bmod{p}$. Hence, there exists a $gH \in X^K$. This implies that for all $k \in K$, $k \cdot gH = gH$. Therefore, $g^{-1}kg \in H$ for all $k \in K$. Since $H$ and $K$ have the same cardinality, we must have that $g^{-1}Kg = H$. So, $H$ and $K$ are conjugate.
    \end{proof}

    \begin{theorem}[Sylow III]
        Let $G$ be a group of order $p^nm$, for a prime $p$ with $(p, m) = 1$. Then, the number of Sylow-$p$ subgroups, $n_p$, satisfies the following:
        \begin{itemize}
            \item $n_p \equiv 1 \bmod{p}$; and
            \item $n_p \mid m$.
        \end{itemize}
    \end{theorem}
    \begin{proof}
        Let $X = \operatorname{Syl}_p(G)$ be the set of all the Sylow-$p$ subgroups in $G$, and fix $H \in X$. Then, $H$ acts on $X$ by conjugation- this is well-defined by Sylow II. We know that $H \in X^H$, so the set of fixed points $X^H$ is not empty. In that case, let $K \in X^H$. We claim that $K = H$. Since $K \in X^H$, we find that for all $h \in H$, $hKh^{-1} = K$. Hence, $H \leq N_G(K)$. Since $N_G(K) \leq G$, we find that $H$ and $K$ are both Sylow-$p$ subgroups in $N_G(K)$ as well. By Sylow II, we can find a $g \in N_G(K)$ such that $H = gKg^{-1}$. Moreover, since $g \in N_G(K)$, we also have $gKg^{-1} = K$. Hence, $H = K$. This implies that $X^H = \{H\}$. Therefore, 
        \[n_p = |X| \equiv |X^H| = 1 \bmod{p}.\]
        
        Now, let $G$ act on $X$ by conjugation. By Sylow II, we know that this action has precisely one orbit, of size $n_p$. So, the Orbit-Stabliser theorem tells us that $n_p \mid p^nm$. Moreover, since $n_p \nmid p$, we find that $n_p \mid m$.
    \end{proof}

    \begin{corollary}
        Let $G$ be a group of order $p^nm$, for a prime $p$ with $(p, m) = 1$. Then, $n_p = [G:N_G(H)]$, where $H$ is a Sylow-$p$ subgroup.
    \end{corollary}
    \begin{proof}
        Let $G$ act on $X$ by conjugation. We have $|\operatorname{Orb}_G(H)| = n_p$. Moreover,
        \[\operatorname{Stab}_G(H) = \{g \in G \mid gHg^{-1} = H\} = N_G(H).\]
        Hence, the Orbit-Stabliser theorem tells us that $n_p = [G:N_G(H)]$.
    \end{proof}

    \begin{lemma}
        Let $G$ be a group of order $p^nm$, for a prime $p$ with $(p, m) = 1$. Then, $n_p = 1$ if and only if there exists a Sylow-$p$ subgroup that is normal in $G$.
    \end{lemma}
    \begin{proof}
        First, assume that $n_p = 1$. Let $H$ be the Sylow-$p$ subgroup, and let $g \in G$. We know that $gHg^{-1}$ is also a Sylow-$p$ subgroup. But, since $n_p = 1$, we find that $gHg^{-1} = H$. Hence, $H$ is normal in $G$.

        Now, assume that $H \leq G$ is a Sylow-$p$ subgroup that is normal in $G$. Let $K$ be a Sylow-$p$ subgroup. By Sylow II, we find that $H$ and $K$ are conjugate. But, since $H$ is normal, we find that $K = H$. So, $n_p = 1$.
    \end{proof}

    \begin{proposition}
        Let $G$ be a finite group, and let $H \leq G$ such that $(|H|, [G:H]) = 1$. Then, $H \vartriangleleft G$ if and only if $H$ is the unique subgroup of order $|H|$.
    \end{proposition}
    \begin{proof}
        First, assume that $H$ is the unique subgroup of order $|H|$, and let $g \in G$. We know that $gHg^{-1}$ is also a subgroup of order $|H|$. Hence, we have $gHg^{-1} = H$. So, $H$ is normal in $G$.

        Now, assume that $H \vartriangleleft G$. Let $K$ be a subgroup of order $|H|$. Consider the restriction of the quotient map $q: K \to G/H$. This is a homomorphism. Moreover, since $|K|$ and $|G/H|$ are coprime, the first isomorphism theorem tells us that $q$ is the trivial homomorphism. Hence, for all $k \in K$, $kH = H$, and so $K \subseteq H$. Since $K$ and $H$ have the same cardinality, this implies that $K = H$. So, $H$ must be the unique subgroup of order $|H|$.
    \end{proof}

    \newpage

    \section{Consequence of Sylow Theorems}
    \begin{proposition}
        A group of order 15 is cyclic.
    \end{proposition}
    \begin{proof}
        Let $G$ be a group of order $15 = 3 \cdot 5$. Let $G$ have $n_3$ Sylow-3 subgroups and $n_5$ Sylow-5 subgroups. By Sylow III, we know that $n_3 \equiv 1 \bmod{3}$ and $n_3 \in \{1, 5\}$, and $n_5 \equiv 1 \bmod{5}$ and $n_5 \in \{1, 3\}$. This implies that both $n_3 = 1$ and $n_5 = 1$. So, let $H$ be the Sylow-$3$ subgroup and $K$ be the Sylow-$5$ subgroup. We know that both $H$ and $K$ are normal in $G$. Moreover, $H \cap K = \{e\}$. This implies that $HK \leq G$ with $HK \cong H \times K$. Since $|H| = 3$ and $|K| = 5$, we find that $G = HK$. Since $H$ and $K$ are cyclic groups of coprime order, we know that $G \cong H \times K$ must be cyclic.
    \end{proof}

    \begin{proposition}
        Let $p$ and $q$ be primes with $p < q$ and $q \not\equiv 1 \bmod{p}$. Then, a group of order $pq$ is cyclic.
    \end{proposition}
    \begin{proof}
        Let $G$ be a group of order $pq$. Let $G$ have $n_p$ Sylow-$p$ subgroups and $n_q$ Sylow-$q$ subgroups. By Sylow III, we know that $n_p \equiv 1 \bmod{p}$ and $n_p \in \{1, q\}$. Since $q \not\equiv 1 \bmod{p}$, we find that $n_p = 1$. Similarly, $n_q \equiv 1 \mod{q}$ and $n_q \in \{1, p\}$. Since $p < q$, we find too that $n_q = 1$. Hence, $G \cong H \times K$ must be cyclic.
    \end{proof}
    
    \begin{proposition}
        A group of order 45 is abelian.
    \end{proposition}
    \begin{proof}
        Let $G$ be a group of order $45 = 3^2 \cdot 5$. Let $G$ have $n_3$ Sylow-3 subgroups and $n_5$ Sylow-5 subgroups. By Sylow III, we know that both $n_3 = 1$ and $n_5 = 1$. So, let $H$ be the Sylow-3 subgroup and $K$ be the Sylow-5 subgroup. We find that $G \cong H \times K$. A group of order $5$ or $9 = 3^2$ must be abelian. Hence, $G$ is abelian.
    \end{proof}

    \begin{proposition}
        A group of order 18 is not simple.
    \end{proposition}
    \begin{proof}
        Let $G$ be a group of order $18 = 2 \cdot 3^2$. Let $G$ have $n_2$ Sylow-2 subgroups and $n_3$ Sylow-3 subgroups. By Sylow III, we know that $n_2 \in \{1, 3, 9\}$ and $n_3 = 1$. In that case, there exists a proper, non-trivial normal subgroup of $G$, with order $9$. Hence, $G$ is not simple.
    \end{proof}

    \begin{proposition}
        A group of order 12 is not simple.
    \end{proposition}
    \begin{proof}
        Let $G$ be a group of order $12 = 2^2 \cdot 3$. Let $G$ have $n_2$ Sylow-2 subgroups and $n_3$ Sylow-3 subgroups. By Sylow III, we know that $n_2 \in \{1, 3\}$ and $n_3 \in \{1, 4\}$. If $n_3 = 1$, then $G$ has a normal Sylow-$3$ subgroup. Otherwise, we have $n_3 = 4$ subgroups of order $3$. In that case, there exist $4 \cdot 2 = 8$ elements of order $3$ in $G$. Since $n_2 \geq 1$, we must have that the remaining $4$ elements form the single Sylow-2 subgroup. So, $n_2 = 1$. Hence, either $n_3 = 1$ or $n_2 = 1$. Therefore, there exists a proper, non-trivial normal subgroup of $G$, with order either $4$ or $3$. Hence, $G$ is not simple.
    \end{proof}
    
    \begin{proposition}
        A group of order 48 is not simple.
    \end{proposition}
    \begin{proof}
        Let $G$ be a group of order $48 = 2^4 \cdot 3$. By Sylow III, we know that $n_3 \in \{1, 3\}$. If $n_3 = 1$, then we have a normal Sylow-$3$ subgroup. Otherwise, we have $n_3 = 3$. Let $H$ and $K$ be two of the Sylow-$3$ subgroups. We know that
        \[|HK| = \frac{|H| |K|}{|H \cap K|} = \frac{2^4 \cdot 2^4}{|H \cap K|} \leq 2^4 \cdot 3.\]
        Since $H \neq K$, we must have $|H \cap K| = 8$. So, we have $[H: H \cap K] = 2 = [K: H \cap K]$. This implies that $H, K \vartriangleleft H \cap K$. Hence, $H, K \leq N_G(H \cap K)$. Therefore, $HK \subseteq N_G(H \cap K)$. We know that
        \[|HK| = \frac{|H| |K|}{|H \cap K|} = 32.\]
        So, Lagrange's theorem tells us that $N_G(H \cap K) = G$. That is, $H \cap K \vartriangleleft G$. This implies that $G$ has a normal subgroup of order $8$. So, $G$ is not simple.
    \end{proof}

    \begin{proposition}
        A group of order 255 is abelian. In particular, it is cyclic.
    \end{proposition}
    \begin{proof}
        Let $G$ be a group of order $255 = 3 \cdot 5 \cdot 17$. By Sylow III, we have $n_{17} = 1$. So, let $H$ be the Sylow-17 subgroup. In that case, $G/H$ is a group of order 15. Since $5 \not\equiv 1 \mod{3}$, we find that $G/H$ is abelian. Hence, the commutator $[G, G] \leq H$. Now, by Sylow III, we have $n_3 \in \{1, 85\}$ and $n_5 \in \{1, 51\}$. If $n_3 = 85$ and $n_5 = 51$, then there are at least
        \[85 \cdot 2 + 51 \cdot 4 = 374\]
        elements in $G$- this is a contradiction. So, we must have either $n_3 = 1$ or $n_5 = 1$. So, let $K$ be the unique Sylow-3 or the Sylow-5 subgroup. Since both $17 \not\equiv 1 \bmod{3}$ and $17 \not\equiv 1 \mod{5}$, we find that $G/K$ is abelian. Hence, $[G, G] \leq K$. So, $[G, G] \leq H \cap K$. By Lagrange, we find that $H \cap K = \{e\}$. This implies that $G$ is abelian.
    \end{proof}

    \begin{proposition}
        A $p$-group is solvable.
    \end{proposition}
    \begin{proof}
        Let $|G| = p^n$, for some prime $p$. By Cauchy's Theorem, we know that $G$ has a subgroup $H_1 \leq G$ of order $p$. By Sylow I, there exists a subgroup $H_2 \leq G$ of order $p^2$ such that $H_2 \vartriangleleft H_1$. We can keep applying Sylow I to find subgroups $H_i \leq G$ of order $p^i$ such that $H_i \vartriangleleft H_{i-1}$, for $2 \leq i \leq n$. Then, the following is a normal series for $G$:
        \[\{e\} = H_0 \vartriangleleft H_1 \vartriangleleft \dots \vartriangleleft H_n = G.\]
        We have $|H_{i+1}/H_i| = p$, so the quotient is abelian. Hence, $G$ is solvable.
    \end{proof}

    \begin{proposition}
        Let $G$ be a finite abelian group and let $m \in \mathbb{Z}_{\geq 1}$ divide $|G|$. Then, $G$ has a subgroup of order $m$.
    \end{proposition}
    \begin{proof}
        Let the prime factorisation of $m$ be:
        \[m = p_1^{\alpha_1} \dots p_n^{\alpha_n}.\]
        By Sylow I, we know that $G$ has a subgroup $P_i \leq G$ of order $p_i^{\alpha_i}$ for all $1 \leq i \leq n$. Since $G$ is abelian, we know that $P_i \vartriangleleft G$. So, we find that 
        \[P = P_1 P_2 \dots P_n\]
        is a (normal) subgroup of $G$. By Lagrange, we find that for $1 \leq i, j \leq n$, if $i \neq j$, then $P_i \cap P_j = \{e\}$. So, it follows that $|P| = p_1^{\alpha_1} \dots p_n^{\alpha_n} = m$.
    \end{proof}
\end{document}