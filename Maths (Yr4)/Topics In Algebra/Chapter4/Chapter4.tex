\documentclass[a4paper, openany]{memoir}

\usepackage[utf8]{inputenc}
\usepackage[T1]{fontenc} 
\usepackage[english]{babel}

\usepackage{fancyhdr}
\usepackage{float}
\usepackage{bm}

\usepackage{amsmath}
\usepackage{amsthm}
\usepackage{amssymb}
\usepackage{enumitem}
\usepackage{multicol}
\usepackage[bookmarksopen=true,bookmarksopenlevel=2]{hyperref}
\usepackage{tikz}
\usepackage{indentfirst}

\pagestyle{fancy}
\fancyhf{}
\fancyhead[LE]{\leftmark}
\fancyhead[RO]{\rightmark}
\fancyhead[RE, LO]{Topics In Algebra}
\fancyfoot[LE, RO]{\thepage}
\fancyfoot[RE, LO]{Pete Gautam}

\renewcommand{\headrulewidth}{1.5pt}

\theoremstyle{definition}
\newtheorem{definition}{Definition}[section]
\newtheorem{example}[definition]{Example}

\theoremstyle{plain}
\newtheorem{theorem}[definition]{Theorem}
\newtheorem{lemma}[definition]{Lemma}
\newtheorem{proposition}[definition]{Proposition}
\newtheorem{corollary}[definition]{Corollary}


\chapterstyle{thatcher}
\setcounter{chapter}{3}

\begin{document}
    \chapter{Free Groups}
    \section{Introduction to Free Groups}
    \begin{definition}
        Let $S$ be a set, and fix a set $S^-$ disjoint to $S$ with a bijection $f\colon S \to S^-$, and a singleton set $\{e\}$. Denote $X_S = S \cup S^- \cup \{1\}$. We define the \emph{inverse map} $-1\colon X_S \to X_S$ by
        \[s^{-1} = \begin{cases}
            e & s = e \\
            \varphi(s) & s \in S \\
            \varphi^{-1}(s) & s \in S^-.
        \end{cases}\]
    \end{definition}

    \begin{definition}
        Let $S$ be a set. A \emph{word} on $S$ is an infinite tuple $(s_1, s_2, \dots)$ with values in $X_S$ such that there exists an $N \in \mathbb{Z}_{\geq 1}$ such that for all $n \in \mathbb{Z}_{\geq 1}$, if $n \geq N$, then $s_n = e$. A \emph{reduced word} on $S$ is a word $(s_1, s_2, \dots)$ such that:
        \begin{itemize}
            \item if $s_N = e$ for some $N \geq 1$, then $s_n = e$ for all $n \geq N$;
            \item if $s_i \neq e$, then $s_{i+1} \neq s_1^{-1}$ for all $n \in \mathbb{Z}_{\geq 1}$.
        \end{itemize}
        We denote a reduced word $(s_1, s_2, \dots, s_n, e, e, \dots)$ by $s_1s_2 \dots s_n$, where $s_n \neq e$. The set of all reduced words is denoted by $F(S)$. We have the inclusion map $\iota \colon S \to F(S)$ given by $\iota(s) = (s, e, e, \dots)$. We also denote $e = (e, e, e, \dots)$, and call it \emph{identity element}.
    \end{definition}

    \begin{definition}
        Let $S$ be a set. Define the operation $\cdot \colon F(S) \to F(S)$ by 
        \[s_1 \dots s_n \cdot t_1 \dots t_k = \]
        The operation is called \emph{concatenation}.
    \end{definition}

    \begin{proposition}
        Let $S$ be a set. Then, $F(S)$ is a group under concatenation.
    \end{proposition}
    \begin{proof}
        
    \end{proof}

    \begin{proposition}[Universal Property of Free Groups]
        Let $S$ be a set, $G$ be a group, and $f: S \to G$ be a map. Then, there exists a unique homomorphism $\varphi: F(S) \to G$ such that $\varphi(s) = f(s)$ for all $s \in S$.
    \end{proposition}
    \begin{proof}
        Define the map $\varphi \colon F(S) \to G$ by 
        \[\varphi(s_1^{\varepsilon_1} s_2^{\varepsilon_2} \dots s_n^{\varepsilon_n}) = f(s_1)^{\varepsilon_1} f(s_2)^{\varepsilon_2} \dots f(s_n)^{\varepsilon_n}.\]
        By construction, this is a group homomorphism. Moreover, it extends $f$.

        Now, let $\psi \colon F(S) \to G$ be such that $\psi(s) = f(s)$ for all $s \in S$. In that case, for all $s_1^{\varepsilon_1} s_2^{\varepsilon_2} \dots s_n^{\varepsilon_n} \in F(S)$, we find that
        \begin{align*}
            \psi(s_1^{\varepsilon_1} s_2^{\varepsilon_2} \dots s_n^{\varepsilon_n}) &= \psi(s_1)^{\varepsilon_1} \psi(s_2)^{\varepsilon_2} \dots \psi(s_n)^{\varepsilon_n} \\
            &= f(s_1)^{\varepsilon_1} f(s_2)^{\varepsilon_2} \dots f(s_n)^{\varepsilon_n} = \varphi(s_1^{\varepsilon_1} s_2^{\varepsilon_2} \dots s_n^{\varepsilon_n}).
        \end{align*}
        So, the map is unique.
    \end{proof}

    \begin{corollary}
        Let $S$ be a set, with free groups $F_1(S)$ and $F_2(S)$. Then, there exists a unique isomorphism $\phi \colon F_1(S) \to F_2(S)$ that fixes $S$.
    \end{corollary}
    \begin{proof}
        Let $\iota_1 \colon S \hookrightarrow F_1(S)$ and $\iota_2 \colon S \hookrightarrow F_2(S)$ be the inclusion maps. We can apply the universal property of the free group $F_2(S)$ on the map $\iota_1$ to extend it to a unique homomorphism $\varphi_1 \colon F_1(S) \to F_2(S)$. Similarly, we can construct a homomorphism $\varphi_2 \colon F_2(S) \to F_1(S)$. Note that, by construction, $\varphi_1$ and $ \varphi_2$ fix $S$. Now, consider the map $\varphi_2 \circ \varphi_1 \colon F_1(S) \to F_1(S)$. This is a group homomorphism that fixes $S$. We can apply again the universal property of the free group $F_1(S)$ on the map $\iota_1$ to extend it to a unique homomorphism $\psi \colon F_1(S) \to F_1(S)$. Note that the identity map is also a homomorphism $\psi \colon F_1(S) \to F_1(S)$, so by uniqueness we find that $\psi = \varphi_2 \circ \varphi_1$ are the identity map on $F_1(S)$. Similarly, $\varphi_1 \circ \varphi_2$ is the identity map on $F_2(S)$. Hence, $\varphi_1$ is an isomorphism with inverse $\varphi_2^{-1}$. By construction, the map is unique and fixes $S$.
    \end{proof}
    
    \begin{definition}
        Let $S$ be a set. We say that $F(S)$ is the \emph{free group} on $S$. We say that $S$ is the set of \emph{free generators} (or \emph{free basis}) of $F(S)$. The \emph{rank} of the free group $F(S)$ is the cardinality of $S$.
    \end{definition}

    \begin{proposition}
        A free group of rank 0 is isomorphic to the trivial group.
    \end{proposition}
    \begin{proof}
        
    \end{proof}
    
    \begin{proposition}
        A free group of rank 1 is isomorphic to $\mathbb{Z}$.
    \end{proposition}
    \begin{proof}
        
    \end{proof}

    \begin{proposition}
        A free group of rank $n \geq 2$ is not abelian.
    \end{proposition}
    \begin{proof}
        
    \end{proof}

    \begin{proposition}
        A free group has no torsion elements.
    \end{proposition}
    \begin{proof}
        
    \end{proof}

    \begin{theorem}[Neilson-Schrier Theorem]
        Let $F$ be a free group and let $G \subseteq F$. Then, $G$ is free.
    \end{theorem}
    
    \newpage

    \section{Group Relations and Presentation}
    \begin{lemma}
        Let $G$ be a group. Then, $G$ is the image of some free group. In particular, there exists a free group $F$ and a surjective group homomorphism $\varphi \colon F \to G$.
    \end{lemma}
    \begin{proof}
        Consider the free group $F(G)$. By the universal propery of free groups on the identity map $id \colon G \to G$, we can extend it to a group homomorphism $\varphi \colon F(G) \to G$. By construction, we know that $\varphi(g) = g$ for all $g \in G$, meaning that $\varphi$ is surjective.
    \end{proof}

    \begin{definition}
        Let $G$ be a group and let $R \subseteq G$. Then, the \emph{normal closure} of $R$ is the intersection of all normal subgroups of $G$ containing $R$. It is denoted by $\langle \langle R \rangle \rangle$.
    \end{definition}

    \begin{proposition}
        Let $G$ be a group and let $R \subseteq G$. Then, $\langle \langle R \rangle \rangle$ is the subgroup generated by the conjugates of $R$.
    \end{proposition}
    \begin{proof}
        Since the normal closure $\langle \langle R \rangle \rangle$ is normal, we know that the conjugates of $R$ are in the subgroup. Moreover, a subgroup generated by the conjugates of $R$ is closed under conjugation by construction, meaning that it is normal, and contains $R$. Hence, it is contained in $\langle\langle R \rangle\rangle$. So, the normal closure is the subgroup generated by the conjugates of $R$.
    \end{proof}

    \begin{proposition}
        Let $G, H$ be groups, $R \subseteq G$ and let $\varphi \colon G \to H$ be a homomorphism with $R \subseteq \ker \varphi$. Then, $\langle \langle R \rangle \rangle \leq \ker \varphi$. In particular, $\langle \langle R \rangle \rangle$ is the smallest unique kernel of a group homomorphism that sends $R$ to the identity.
    \end{proposition}
    \begin{proof}
        Since $\ker \varphi$ is a normal subgroup, and $R \subseteq \ker \varphi$, it follows that $\langle \langle R \rangle \rangle \leq \ker \varphi$.
    \end{proof}

    \begin{definition}
        Let $G$ be a group and $S$ a generating set of $G$. A \emph{presentation} is a pair $(S, R)$, where $R$ is a set of words in $F(S)$ such that the normal closure $\langle \langle R \rangle \rangle$ is the kernel of the homomorphism $\varphi \colon F(S) \to G$ that fixes $S$. The set $R$ is called the \emph{relators}. We denote $G = \langle S \mid R \rangle$. 
        
        We say that $G$ is \emph{finitely presented} if there exists a presentation of $G$, $(S, R)$, such that both $S$ and $R$ are finite. We say that $G$ is \emph{finitely generated} if there exists a presentation of $G$, $(S, R)$, such that $S$ is finite.
    \end{definition}

    \begin{proposition}
        Let $G$ be a finite group. Then, $G$ is finitely presented.
    \end{proposition}
    \begin{proof}
        
    \end{proof}

    \begin{proposition}
        Let $G$ and $H$ be groups with bijective presentations. Then, there exists a group isomorphism $G \to H$.
    \end{proposition}
    \begin{proof}
        
    \end{proof}

    \begin{proposition}
        There is one non-abelian group of order 10 up to isomorphism.
    \end{proposition}
    \begin{proof}
        
    \end{proof}

\end{document}