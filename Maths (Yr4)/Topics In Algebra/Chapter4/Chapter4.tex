\documentclass[a4paper, openany]{memoir}

\usepackage[utf8]{inputenc}
\usepackage[T1]{fontenc} 
\usepackage[english]{babel}

\usepackage{fancyhdr}
\usepackage{float}
\usepackage{bm}

\usepackage{amsmath}
\usepackage{amsthm}
\usepackage{amssymb}
\usepackage{enumitem}
\usepackage{multicol}
\usepackage[bookmarksopen=true,bookmarksopenlevel=2]{hyperref}
\usepackage{tikz}
\usepackage{indentfirst}

\pagestyle{fancy}
\fancyhf{}
\fancyhead[LE]{\leftmark}
\fancyhead[RO]{\rightmark}
\fancyhead[RE, LO]{Topics In Algebra}
\fancyfoot[LE, RO]{\thepage}
\fancyfoot[RE, LO]{Pete Gautam}

\renewcommand{\headrulewidth}{1.5pt}

\theoremstyle{definition}
\newtheorem{definition}{Definition}[section]
\newtheorem{example}[definition]{Example}

\theoremstyle{plain}
\newtheorem{theorem}[definition]{Theorem}
\newtheorem{lemma}[definition]{Lemma}
\newtheorem{proposition}[definition]{Proposition}
\newtheorem{corollary}[definition]{Corollary}


\chapterstyle{thatcher}
\setcounter{chapter}{3}

\begin{document}
    \chapter{Free Groups}
    \section{Introduction to Free Groups}
    \begin{definition}
        Let $S$ be a set, and fix a set $S^-$ disjoint to $S$ with a bijection $f\colon S \to S^-$, and a singleton set $\{e\}$ disjoint to $S \cup S^-$. Denote $X_S = S \cup S^- \cup \{e\}$. We define the \emph{inverse map} $-1\colon X_S \to X_S$ by
        \[s^{-1} = \begin{cases}
            e & s = e \\
            f(s) & s \in S \\
            f^{-1}(s) & s \in S^-.
        \end{cases}\]
    \end{definition}

    \begin{definition}
        Let $S$ be a set. A \emph{word} on $S$ is an infinite tuple $(s_1, s_2, \dots)$ with values in $X_S$ such that there exists an $N \in \mathbb{Z}_{\geq 1}$ such that for all $n \in \mathbb{Z}_{\geq 1}$, if $n \geq N$, then $s_n = e$. A \emph{reduced word} on $S$ is a word $(s_1, s_2, \dots)$ such that:
        \begin{itemize}
            \item if $s_N = e$ for some $N \geq 1$, then $s_n = e$ for all $n \geq N$;
            \item if $s_n \neq e$, then $s_{n+1} \neq s_n^{-1}$ for all $n \in \mathbb{Z}_{\geq 1}$.
        \end{itemize}
        We denote a reduced word $(s_1, s_2, \dots, s_n, e, e, \dots)$ by $s_1s_2 \dots s_n$, where $s_n \neq e$. The set of all reduced words is denoted by $F(S)$. We also denote $e = (e, e, e, \dots)$, and call it \emph{empty word}.
    \end{definition}

    \begin{definition}
        Let $S$ be a set. Define the operation $\cdot \colon F(S) \to F(S)$ as follows: let $s_1 \dots s_m, t_1 \dots t_n \in F(S)$, with $m \leq n$. Fix $k \in \mathbb{Z}$ such that $s_{m-k+1} \neq t_k^{-1}$. Then,
        \[(s_1 \dots s_m) \cdot (t_1 \dots t_n) = \begin{cases}
            s_1 \dots s_{m-k+1} t_k \dots t_n & k \leq m \\
            t_{m+1} \dots t_n & k = m+1 \leq n \\
            1 & k = m+1, m = n
        \end{cases}\]
        We can similarly define the operation if $m \geq n$. The operation is called \emph{concatenation}.
    \end{definition}

    \begin{proposition}
        Let $S$ be a set. Then, $F(S)$ is a group under concatenation.
    \end{proposition}
    \begin{proof}
        By definition, we have an identity element, and every element in $F(S)$ has an inverse:
        \[(s_1 \dots s_m)^{-1} = s_m^{-1} \dots s_1^{-1}.\]
        We now show that the operation is associative. First, for $s \in X_S$, define the map $\sigma_s \colon F(S) \to F(S)$ by:
        \[\sigma_s(s_1 \dots s_n) = \begin{cases}
            s s_1 \dots s_n & s^{-1} \neq s_1 \\
            s_2 \dots s_n & \textrm{otherwise}.
        \end{cases}\]
        We note that $\sigma_s \circ \sigma_{s^{-1}} = id$, meaning that $\sigma_s$ is a bijection. Hence, $\sigma_s \in \operatorname{Perm} (F(S))$. So, define
        \[B(S) = \langle \sigma_s \mid s \in X_S \rangle \leq \operatorname{Perm} (F(S)),\]
        and the map $f \colon F(S) \to B(S)$ by
        \[f(s_1 \dots s_n) = \sigma_{s_1} \dots \sigma_{s_n}.\]
        Since $f$ obeys concatenation in $B(S)$, and $B(S)$ is associative, it follows that concatenation is associative. Hence, $F(S)$ is a group.
    \end{proof}

    \begin{proposition}[Universal Property of Free Groups]
        Let $S$ be a set, $G$ be a group, and $f: S \to G$ be a map. Then, there exists a unique homomorphism $\varphi: F(S) \to G$ such that $\varphi(s) = f(s)$ for all $s \in S$.
    \end{proposition}
    \begin{proof}
        Define the map $\varphi \colon F(S) \to G$ by 
        \[\varphi(s_1 s_2 \dots s_n) = f(s_1) f(s_2) \dots f(s_n).\]
        By construction, this is a group homomorphism. Moreover, it extends $f$ by definition.

        Now, let $\psi \colon F(S) \to G$ be such that $\psi(s) = f(s)$ for all $s \in S$. In that case, for all $s_1 s_2 \dots s_n \in F(S)$, we find that
        \begin{align*}
            \psi(s_1 s_2 \dots s_n) &= \psi(s_1) \psi(s_2) \dots \psi(s_n) \\
            &= f(s_1) f(s_2) \dots f(s_n) \\
            &= \varphi(s_1 s_2 \dots s_n).
        \end{align*}
        So, the map is unique.
    \end{proof}

    \begin{corollary}
        Let $S$ be a set, with free groups $F_1(S)$ and $F_2(S)$. Then, there exists a unique isomorphism $\phi \colon F_1(S) \to F_2(S)$ that fixes $S$.
    \end{corollary}
    \begin{proof}
        Let $\iota_1 \colon S \hookrightarrow F_1(S)$ and $\iota_2 \colon S \hookrightarrow F_2(S)$ be the inclusion maps. We can apply the universal property of the free group $F_2(S)$ on the map $\iota_1$ to extend it to a unique homomorphism $\varphi_1 \colon F_1(S) \to F_2(S)$. Similarly, we can construct a homomorphism $\varphi_2 \colon F_2(S) \to F_1(S)$. Note that, by construction, $\varphi_1$ and $ \varphi_2$ fix $S$. Now, consider the map $\varphi_2 \circ \varphi_1 \colon F_1(S) \to F_1(S)$. This is a group homomorphism that fixes $S$. We can apply again the universal property of the free group $F_1(S)$ on the map $\iota_1$ to extend it to a unique homomorphism $\psi \colon F_1(S) \to F_1(S)$. Note that the identity map is also a homomorphism $\psi \colon F_1(S) \to F_1(S)$, so by uniqueness we find that $\psi = \varphi_2 \circ \varphi_1$ are the identity map on $F_1(S)$. Similarly, $\varphi_1 \circ \varphi_2$ is the identity map on $F_2(S)$. Hence, $\varphi_1$ is an isomorphism with inverse $\varphi_2^{-1}$. By construction, the map is unique and fixes $S$.
    \end{proof}
    
    \begin{definition}
        Let $S$ be a set. We say that $F(S)$ is the \emph{free group} on $S$. We say that $S$ is the set of \emph{free generators} (or \emph{free basis}) of $F(S)$. The \emph{rank} of the free group $F(S)$ is the cardinality of $S$.
    \end{definition}

    \begin{proposition}
        A free group of rank 0 is isomorphic to the trivial group.
    \end{proposition}
    \begin{proof}
        Let $G$ be a free group of rank 0. In that case, $G$ is the free group on $\varnothing$. Hence, it has precisely one element- the identity.
    \end{proof}
    
    \begin{proposition}
        A free group of rank 1 is isomorphic to $\mathbb{Z}$.
    \end{proposition}
    \begin{proof}
        Let $F$ be a free group of rank $1$, and let $F$ be generated by some $x \in F$. In that case, we know that every $y \in F$ is of the form $y = x^n$, and so $y \in \langle x \rangle$. Hence, $F = \langle x \rangle$. Since $x$ has infinite order, it follows that $F$ is isomorphic to $\mathbb{Z}$.
    \end{proof}

    \begin{proposition}
        A free group of rank $n \geq 2$ is not abelian.
    \end{proposition}
    \begin{proof}
        Let $F$ be a free group of rank $n$. Consider distinct elements $a, b \in F$ that generate $F$. In that case, we know that in $F(S)$, $ab \neq ba$. Hence, $F(S)$ is not abelian.
    \end{proof}

    \begin{proposition}
        A free group has no torsion elements.
    \end{proposition}
    \begin{proof}
        Let $F$ be a free group, and let $x \in F$ be non-trivial.
        % TODO: Contradiction?
    \end{proof}

    \begin{theorem}[Neilson-Schrier Theorem]
        Let $F$ be a free group and let $G \leq F$. Then, $G$ is free.
    \end{theorem}
    
    \newpage

    \section{Group Relations and Presentation}
    \begin{lemma}
        Let $G$ be a group. Then, $G$ is the image of some free group. In particular, there exists a free group $F$ and a surjective group homomorphism $\varphi \colon F \to G$.
    \end{lemma}
    \begin{proof}
        Consider the free group $F(G)$. By the universal propery of free groups on the identity map $id \colon G \to G$, we can extend it to a group homomorphism $\varphi \colon F(G) \to G$. By construction, we know that $\varphi(g) = g$ for all $g \in G$, meaning that $\varphi$ is surjective.
    \end{proof}

    \begin{definition}
        Let $G$ be a group and let $R \subseteq G$. Then, the \emph{normal closure} of $R$ is the intersection of all normal subgroups of $G$ containing $R$. It is denoted by $\langle \langle R \rangle \rangle$.
    \end{definition}

    \begin{proposition}
        Let $G$ be a group and let $R \subseteq G$. Then, $\langle \langle R \rangle \rangle$ is the subgroup generated by the conjugates of $R$.
    \end{proposition}
    \begin{proof}
        Since the normal closure $\langle \langle R \rangle \rangle$ is normal, we know that the conjugates of $R$ are in the subgroup. Moreover, a subgroup generated by the conjugates of $R$ is closed under conjugation by construction, meaning that it is normal, and contains $R$. Hence, it is contained in $\langle\langle R \rangle\rangle$. So, the normal closure is the subgroup generated by the conjugates of $R$.
    \end{proof}

    \begin{proposition}
        Let $G, H$ be groups, $R \subseteq G$ and let $\varphi \colon G \to H$ be a homomorphism with $R \subseteq \ker \varphi$. Then, $\langle \langle R \rangle \rangle \leq \ker \varphi$. In particular, $\langle \langle R \rangle \rangle$ is the smallest unique kernel of a group homomorphism that sends $R$ to the identity.
    \end{proposition}
    \begin{proof}
        Since $\ker \varphi$ is a normal subgroup, and $R \subseteq \ker \varphi$, it follows that $\langle \langle R \rangle \rangle \leq \ker \varphi$.
    \end{proof}

    \begin{definition}
        Let $G$ be a group and $S$ a generating set of $G$. A \emph{presentation} is a pair $(S, R)$, where $R$ is a set of words in $F(S)$ such that the normal closure $\langle \langle R \rangle \rangle$ is the kernel of the homomorphism $\varphi \colon F(S) \to G$ that fixes $S$. The set $R$ is called the \emph{relators}. We denote $G = \langle S \mid R \rangle$. 
        
        We say that $G$ is \emph{finitely presented} if there exists a presentation of $G$, $(S, R)$, such that both $S$ and $R$ are finite. We say that $G$ is \emph{finitely generated} if there exists a presentation of $G$, $(S, R)$, such that $S$ is finite.
    \end{definition}

    \begin{proposition}
        Let $G$ be a finite group. Then, $G$ is finitely presented.
    \end{proposition}
    \begin{proof}
        Let $S = G$, and define the set
        \[R = \{ghk^{-1} \in F(S) \mid g, h \in G, gh = h\}.\]
        Since $G$ is finite, it follows that $R$ is finite. We claim that $G = \langle S \mid R \rangle$. 
        
        Let $N$ be the normal closure of $R$ in $F(S)$. Consider the group $H = F(S)/N$. We know that there is an extension of the inclusion map $\psi \colon F(S) \to G$- this follows from the universal property of free groups. Moreover, $N \leq \ker \psi$, so the universal property of the quotient gives us a map $\varphi \colon H \to G$. By construction, we find that $\{gN \mid g \in G\}$ generates $H$. Since $G$ is a group, it is also closed under the binary operation. Hence, 
        \[H = \{gN \mid g \in G\}.\]
        In particular, $|H| = |G|$. Hence, we find that $N = \ker \psi$. This implies that $\langle S \mid R \rangle$ is a presentation for $G$. So, $G$ is finitely presented.
    \end{proof}

    \begin{proposition}
        Let $G$ be a group with the presentation
        \[G = \langle a, b \mid r_1, \dots, r_k \rangle.\]
        Then, for a group $H$ generated by two elements $x, y \in H$ that satisfy the relations $r_1, \dots, r_k$, there exists a group homomorphism $\Phi \colon G \to H$.
    \end{proposition}
    \begin{proof}
        Consider the presentation homomorphism $\varphi \colon F(a, b) \to G$. This is surjective since it extends the identity map on $\{a, b\}$. Now, consider the map $f \colon \{a, b\} \to H$ given by $f(a) = x$ and $f(b) = y$. By the universal property of free groups, we can extend $f$ to a group homomorphism $\psi \colon F(a, b) \to H$. We know that $H$ satisfies the relations $r_1, \dots, r_k$, so we have $\ker \varphi \leq \ker \psi$. So, the universal property of quotients tells us that there exists a group homomorphism $\Phi \colon G \to H$.
    \end{proof}

    \begin{proposition}
        There is one non-abelian group of order 10 up to isomorphism.
    \end{proposition}
    \begin{proof}
        Let $G$ be a non-abelian group of order 10. By Cauchy's Theorem, we can find a $g \in G$ of order $5$. Set $H = \langle g \rangle$. We know that $H$ has index $2$ in $G$, so it is a normal subgroup of $G$. Now, let $k \in G$ such that $k \not\in H$. We know that $kH \neq H$, so $kH$ must have order 2. Hence, $k^2 \in H$. If $|k^2| = 5$, then $|k| = 10$, meaning that $G$ is cyclic. This cannot be the case, so we must have $k^2 = e$. Since $H$ is normal in $G$, we find that $kgk^{-1} \in H$. Since $G$ is not abelian, $k$ and $g$ cannot commute. Hence, $kgk^{-1} \in \{g^2, g^3, g^4\}$. We consider each case separately:
        \begin{itemize}
            \item First, assume that $kgk^{-1} = g^2$. In that case, 
            \begin{align*}
                g &= eg \\
                &= k^2g \\
                &= k \cdot kg \\
                &= k \cdot g^2k \\
                &= kg \cdot gk \\
                &= g^2k \cdot gk \\
                &= g^2 \cdot kg \cdot k \\
                &= g^2 \cdot g^2k \cdot k \\
                &= g^4 k^2 = g^4.
            \end{align*}
            Hence, $g^3 = e$. Since $g$ has order $5$, this is a contradiction.

            \item Now, assume that $kgk^{-1} = g^3$. In that case,
            \begin{align*}
                g &= k \cdot kg \\
                &= k \cdot g^3k \\
                &= kg \cdot g^2 k \\
                &= g^3k \cdot g^2 k \\
                &= g^3 \cdot kg \cdot gk \\
                &= g^3 \cdot g^3k \cdot gk  \\
                &= g \cdot kg \cdot k \\
                &= g \cdot g^3k \cdot k \\
                &= g^4 k^2 = g^4.
            \end{align*}
            Like above, this is a contradiction.
            
            \item Finally, assume that $kgk^{-1} = g^4 = g^{-1}$. In that case, the presentation for $G$ is:
            \[G = \langle g, k \mid g^5 = e = k^2, kgk^{-1} = g^{-1}\rangle.\]
            We know that
            \[D_5 = \langle r, s \mid r^5 = e = s^2, rs = sr^{-1} \rangle.\]
            So, $D_4$ and $G$ have isomophic presentations, meaning that $G \cong D_5$.
        \end{itemize}
        Hence, there is only one non-abelian group of order $10$.
    \end{proof}

\end{document}