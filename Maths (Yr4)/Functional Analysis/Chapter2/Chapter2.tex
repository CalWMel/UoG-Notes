\documentclass[a4paper, openany]{memoir}

\usepackage[utf8]{inputenc}
\usepackage[T1]{fontenc} 
\usepackage[english]{babel}

\usepackage{fancyhdr}
\usepackage{float}
\usepackage{bm}

\usepackage{amsmath}
\usepackage{amsthm}
\usepackage{amssymb}
\usepackage{enumitem}
\usepackage{multicol}
\usepackage[bookmarksopen=true,bookmarksopenlevel=2]{hyperref}
\usepackage{tikz}
\usepackage{indentfirst}

\pagestyle{fancy}
\fancyhf{}
\fancyhead[LE]{\leftmark}
\fancyhead[RO]{\rightmark}
\fancyhead[RE, LO]{Functional Analysis}
\fancyfoot[LE, RO]{\thepage}
\fancyfoot[RE, LO]{Pete Gautam}

\renewcommand{\headrulewidth}{1.5pt}

\theoremstyle{definition}
\newtheorem{definition}{Definition}[section]

\theoremstyle{plain}
\newtheorem{theorem}[definition]{Theorem}
\newtheorem{lemma}[definition]{Lemma}
\newtheorem{proposition}[definition]{Proposition}
\newtheorem{corollary}[definition]{Corollary}
\newtheorem{example}[definition]{Example}

\chapterstyle{thatcher}
\setcounter{chapter}{1}

\begin{document}
    \chapter{Measure Theory}
    \section{Rings and Algebras}
    In this section, we will define measures and see why they are necessary. In relation to functional analysis, measure theory is needed to formally define function spaces that complement the sequence spaces $\ell^p$. We will be able to extend $C[0, 1]$ to a bigger set of functions by defining integration for more general spaces.

    Intuitively, a measure on $\mathbb{R}$ is a function $\mu \colon \mathbb{P}(\mathbb{R}) \to [0, \infty]$ such that
    \begin{itemize}
        \item for disjoint collection of countable sets $(A_i)_{i \in I}$,
        \[\mu \left(\bigcup_{i \in I} A_i\right) = \sum_{i \in I} \mu(A_i);\]
        \item for any $x \in \mathbb{R}$ and $A \subseteq \mathbb{R}$, $\mu(x + A) = \mu(A)$ (translation-invariant); and
        \item the measure of the unit interval $\mu([0, 1]) = 1$.
    \end{itemize}
    It turns out that it is not possible to find such a function. We will prove this by showing that satisfying the first 2 axioms implies that the third axiom cannot be satisfied.

    So, assume that we have a measure function $\mu$ on $\mathbb{R}$ satisfying the first two axioms. Then, define the equivalence relation $\sim$ on $[0, 1]$ by 
    \[x \sim y \iff x - y \in \mathbb{Q}.\]
    From each equivalence class, choose a specific element $x$, and let $N$ be the set containing all such elements. Now, for $r \in \mathbb{Q} \cap [0, 1)$, define
    \[N_r = \{x + r \mid x \in N, x \leq 1 - r\} \cup \{x + r - 1 \mid x \in N, x > 1 - r\}.\]
    We know that $N_r \cap N_q \neq \varnothing$ if and only if $r = q$, with
    \[\bigcup_{r \in \mathbb{Q} \cap [0, 1)} N_r = [0, 1).\]
    Since the measure function $\mu$ is translation-invariant, we know that $\mu(N_r) = N$ for all $r \in \mathbb{Q} \cap [0, 1]$. Moreover, since $\mathbb{Q} \cap [0, 1]$ is countable with $N_r$ disjoint for all $r \in \mathbb{Q} \cap [0, 1]$, we find that
    \[\mu([0, 1)) = \mu \left(\bigcup_{r \in \mathbb{Q} \cap [0, 1)}\right) = \sum_{r \in \mathbb{Q} \cap [0, 1)} \mu(N_r) = \sum_{r \in \mathbb{Q} \cap [0, 1)} \mu(N).\]
    The value $\mu(N)$ is a constant, so either $\mu(N) = 0$, in which case $\mu([0, 1)) = 0$, or $\mu(N) > 0$, in which case $\mu([0, 1)) = \infty$. We would like to give the interval $[0, 1)$ a non-zero finite value (in particular 1), and to do so, we cannot allow every subset of $\mathbb{R}$ to be measurable.

    We will now define the theory of rings and algebras, and later use this to define measurable functions, including the one we want in $\mathbb{R}^n$.
    \begin{definition}
        Let $X$ be a set and let $\mathcal{A} \subseteq \mathbb{P}(X)$ be a non-empty collection of subsets of $X$. 
        \begin{itemize}
            \item We say that $(X, \mathcal{A})$ is a \emph{ring} if for all $A, B \in \mathcal{A}$, $A \cup B \in \mathcal{A}$ and $A \setminus B \in \mathcal{A}$.
            \item We say that $(X, \mathcal{A})$ is an \emph{algebra} if for all $A, B \in \mathcal{A}$, $A \cup B \in \mathcal{A}$ and $A^c \in \mathcal{A}$.
            \item We say that $(X, \mathcal{A})$ is a \emph{$\sigma$-algebra} if for all $A \in \mathcal{A}$, $A^c \in \mathcal{A}$, and for a collection $(A_i)_{i=1}^\infty$ of sets in $\mathcal{A}$, the union
            \[\bigcup_{i=1}^\infty A_i \in \mathcal{A}.\]
        \end{itemize}
    \end{definition}
    We will only be focusing on $\sigma$-algebras, but many of the results will hold for rings or algebras. A $\sigma$-algebra is an algebra, and an algebra is a ring. An example of a $\sigma$-algebra is $\mathcal{P}(X)$, for some set $X$. For any set $A \subseteq X$, the smallest $\sigma$-algebra containing $A$ is denoted $\sigma(A)$. It is given by the intersection of $\sigma$-algebras containing $A$- this is always a $\sigma$-algebra. For a topological space $X$, the set $\mathcal{B}(X)$ denotes the Borel sets, which is the $\sigma$-algebra generated by open sets in $X$.

    We want to define measurable functions on $\sigma$-algebras. This can be defined on a ring.
    \begin{definition}
        Let $(X, \mathcal{A})$ be a ring, and let $\mu \colon \mathcal{A} \to [0, \infty]$ be a function.
        \begin{itemize}
            \item We say that $\mu$ is \emph{additive} if for $A, B \in \mathcal{A}$ disjoint, $\mu(A \cup B) = \mu(A) + \mu(B)$.
            \item We say that $\mu$ is \emph{$\sigma$-subadditive} if for a collection of disjoint sets $(A_i)_{i=1}^\infty$,
            \[\mu \left(\bigcup_{i=1}^\infty A_i\right) \leq \sum_{i=1}^\infty \mu(A_i)\]
            if the set $\bigcup_{i=1}^\infty A_i \in \mathcal{A}$.
            \item We say that $\mu$ is \emph{$\sigma$-additive} if for a collection of disjoint sets $(A_i)_{i=1}^\infty$,
            \[\mu \left(\bigcup_{i=1}^\infty A_i\right) = \sum_{i=1}^\infty \mu(A_i)\]
            if the set $\bigcup_{i=1}^\infty A_i \in \mathcal{A}$.
        \end{itemize}
    \end{definition}
    An example of a measurable function on the natural numbers $\mathbb{Z}_{\geq 1}$ is the counting measure i.e. $m(A) = |A|$. In this case, it is possible to define the measure on the entire set. We will later see that integration defined on this set will precisely give us the $\ell^p$ sequence spaces.

    Now, let $\mathcal{E}(\mathbb{R})$ be the set of finite union of intervals in $\mathbb{R}$. This is a ring- the union of two unions is still a union, and the set difference of two intervals is a union of intervals. However, it is not an algebra since it does not contain countable union of intervals. We can define the following measure $\mu$ on $\mathcal{E}(\mathbb{R})$. First, for an interval $I_k$, we define
    \[\mu(I_k) = \begin{cases}
        \infty & I_k \textrm{ not bounded} \\
        \sup I_k - \inf I_k & \textrm{otherwise}.
    \end{cases}\]
    Now, for a union of intervals
    \[I = \bigcup_{k=1}^n I_k,\]
    we can make it a disjoint union of intervals
    \[I = \bigcup_{k=1}^n J_k\]
    (a possible choice is $J_1 = I_1$, $J_2 = I_2 \setminus J_1$, $J_3 = I_3 \setminus (J_1 \cup J_2)$ and so on). Then, we define
    \[\mu(I) = \sum_{k=1}^n \mu(J_k).\]
    By definition, $\mu$ is additive. It is also $\sigma$-additive. We will extend $\mu$ to the $\sigma$-algebra generated by $\mathcal{E}(\mathbb{R})$ in the next section.
    \newpage

    \section{Outer Measure}
    In this section, we will define the concept of outer measure that allows us to extend a measure on a ring to the $\sigma$-algebra generated by the ring. First, we define the outer measure on every subset:
    \begin{definition}
        Let $X$ be a set and let $\mathbb{R}$ be a ring on $X$ and $\mu$ a measure on $\mathcal{R}$. Define the map $\mu^* \colon \mathbb{P}(X) \to [0, \infty]$ by
        \[\mu^*(A) = \inf \left\{\sum_{k=1}^\infty m(I_k) \mid (I_k) \textrm{ collection in } \mathcal{R} \textrm{ s.t. } A \subseteq \bigcup_{i=1}^\infty I_k \right\}\]
    \end{definition}
    \noindent It turns out that $\mu*$ is not a measure on $\mathbb{P}(X)$, but we can restrict it to a measure on the $\sigma$-algebra generated by the ring. 
    
    We first show some properties about $\mu^*$:
    \begin{proposition}
        Let $X$ be a set, $\mathcal{R}$ a ring on $X$ and $\mu$ a measure on $\mathcal{R}$. Then,
        \begin{itemize}
            \item for $A \subseteq B \subseteq X$, $\mu^*(A) \leq \mu^*(B)$;
            \item $\mu^*$ is $\sigma$-subadditive, i.e. for a collection $(A_k)_{k=1}^\infty$ of subsets of $X$,
            \[\mu^*\left(\bigcup_{k=1}^\infty A_k\right) \leq \sum_{k=1}^\infty \mu^*(A_k).\]
            \item $\mu^*$ extends $\mu$, i.e. for $A \in \mathcal{R}$, $\mu^*(A) = \mu(A)$.
        \end{itemize}
    \end{proposition}
    \begin{proof}
        \hspace*{0pt}
         \begin{itemize}
            \item Let $A \subseteq B \subseteq X$. Then, if $(I_k)_{k=1}^\infty$ is a collection in $\mathcal{R}$ such that 
            \[B \subseteq \bigcup_{k=1}^\infty I_k,\]
            then 
            \[A \subseteq \bigcup_{k=1}^\infty I_k.\]
            So, by the definition of the infimum, we find that $\mu^*(A) \leq \mu^*(B)$.

            \item Let $\varepsilon > 0$ and set
            \[A = \bigcup_{k=1}^\infty A_k.\]
            For $k \in \mathbb{Z}_{\geq 1}$, by the definition of $\mu^*(A_k)$, we can find a collection $(I^{(k)}_n)_{n=1}^\infty$ such that
            \[\mu^*(A_k) + \frac{\varepsilon}{2^{k+1}} \geq \sum_{n=1}^\infty \mu(I^{(k)}_n).\]
            Then, we know that
            \[A \subseteq \bigcup_{k=1}^\infty \bigcup_{n=1}^\infty I^{(k)}_n,\]
            with
            \begin{align*}
                \mu^*(A) &\leq \sum_{k=1}^\infty \sum_{k=1}^\infty \mu(I^{(k)}_n) \\
                &\leq \sum_{k=1}^\infty \mu^*(A_k) + \frac{\varepsilon}{2^{k+1}} \\
                &= \varepsilon + \sum_{k=1}^\infty \mu^*(A_k).
            \end{align*}
            So,
            \[\mu^*(A) \leq \sum_{k=1}^\infty \mu^*(A_k).\]
            
            \item 
         \end{itemize}
    \end{proof}

    As we mentioned before, to make $\mu^*$ a $\sigma$-algebra, we need to restrict it to the $\sigma$-algebra generated by the ring. For example, for the set $\mathbb{R}$ with the norm measure on the ring $\mathcal{E}(\mathbb{R})$, we can generate the Lebesgue measure $\lambda^*$ on the $\sigma$-algebra generated by the ring, which coincides with the Borel $\sigma$-algebra, $\mathcal{B}(\mathbb{R})$, which is the $\sigma$-algebra generated by the open sets in $\mathbb{R}$. Note that the Lebesgue-measurable sets $\mathcal{L}(\mathbb{R})$ is the Borel sets $\mathcal{B}(\mathbb{R})$ with union those that have measure $0$ in $\mathbb{R}$, i.e.
    \[\mathcal{L}(\mathbb{R}) = \{B \cup N \mid B \in \mathcal{B}(\mathbb{R}), \mu^*(N) = 0\}.\]
    \newpage

    \section{Measurable Functions}
    In this section, we will consider measurable functions- these will be equivalent to integrable functions, but will also allow the integral to be infinite.
    \begin{definition}
        Let $(X, \mathcal{A}, \mu)$ and $(X, \mathcal{B}, \nu)$ be measure spaces, and let $f \colon X \to Y$ be a function. Then, $f$ is \emph{measurable} if $f^{-1}(B) \in \mathcal{A}$ for all $B \in \mathcal{B}$.
    \end{definition}
    \noindent If $X = Y = \mathbb{R}$, then this definition is equivalent to asking that $f^{-1}(-\infty, a)$ is measurable for all $a \in \mathbb{R}$. We can even generalise this to other similar preimages.
    \begin{proposition}
        Let $f \colon \mathbb{R} \to \mathbb{R}$ be a function. Then, the following are equivalent:
        \begin{enumerate}
            \item $f^{-1}(-\infty, a)$ is measurable for all $a \in \mathbb{R}$
            \item $f^{-1}(a, \infty)$ is measurable for all $a \in \mathbb{R}$
            \item $f^{-1}(-\infty, a]$ is measurable for all $a \in \mathbb{R}$
            \item $f^{-1}[a, \infty)$ is measurable for all $a \in \mathbb{R}$
        \end{enumerate}
    \end{proposition}
    \begin{proof}
        Since measurable sets form a $\sigma$-algebra, we know that it is closed under complements. Hence, we know that $f^{-1}(-\infty, a)$ is measurable if and only if $f^{-1}[a, \infty)$ is measurable, and similarly $f^{-1}(-\infty, a]$ is measurable if and only if $f^{-1}(a, \infty)$ is measurable. So, we show $(1) \implies (3)$ and $(3) \implies (1)$.
        \begin{enumerate}
            \item[$(1) \implies (3)$] Assume that for all $a \in \mathbb{R}$, $f^{-1}(-\infty, a)$ is measurable. Now, for $a \in \mathbb{R}$, we claim that
            \[(-\infty, a] = \bigcap_{n=1}^\infty (-\infty, a + 1/n)\]
            We have $(-\infty, a) \subseteq (-\infty, a + 1/n)$ for all $n \in \mathbb{Z}_{\geq 1}$, meaning that
            \[(-\infty, a] \subseteq \bigcap_{n=1}^\infty (-\infty, a + 1/n).\]
            Next, let $x \in (a, \infty)$. Define $\varepsilon = x - a > 0$. We can find an $n \in \mathbb{Z}_{\geq 1}$ such that $n > \frac{1}{\varepsilon}$. Hence, 
            \[a + \frac{1}{n} < a + \varepsilon = x,\]
            meaning that $x \not\in (-\infty, a + 1/n)$. This implies that
            \[x \not\in \bigcap_{n=1}^\infty (-\infty, a + 1/n).\]
            Hence,
            \[(-\infty, a] = \bigcap_{n=1}^\infty (-\infty, a + 1/n).\]
            So, 
            \[f^{-1}(-\infty, a] = f^{-1} \left(\bigcap_{n=1}^\infty (-\infty, a + 1/n)\right) = \bigcap_{n=1}^\infty f^{-1} (-\infty, a + 1/n).\]
            We know that $(-\infty, a+1/n)$ is measurable for all $n \in \mathbb{Z}_{\geq 1}$. Since a $\sigma$-algebra is closed under countable intersection, it follows that $f^{-1}(-\infty, a]$ is measurable.
            
            \item[$(3) \implies (1)$] Now, assume that for all $a \in \mathbb{R}$, $f^{-1} (-\infty, a]$ is measurable. Then, for $a \in \mathbb{R}$, we claim that
            \[(-\infty, a) = \bigcup_{n=1}^\infty (-\infty, a - 1/n].\]
            We know that $(-\infty, a-1/n] \subseteq (-\infty, a)$ for all $a \in \mathbb{R}$, meaning that
            \[(-\infty, a) \supseteq \bigcup_{n=1}^\infty (-\infty, a - 1/n].\]
            Now, let $x \in (-\infty, a)$. Let $\varepsilon = a - x > 0$. We can find an $n \in \mathbb{Z}_{\geq 1}$ such that $n > \frac{1}{\varepsilon}$. In that case,
            \[a - \frac{1}{n} > a - \varepsilon = x,\]
            which implies that $x \in (-\infty, a - 1/n]$. Hence,
            \[x \in \bigcup_{n=1}^\infty (-\infty, a - 1/n].\]
            So,
            \[(-\infty, a) = \bigcup_{n=1}^\infty (-\infty, a - 1/n].\]
            Using the same argument as above, we find that $f^{-1}(-\infty, a)$ is measurable.
        \end{enumerate}
    \end{proof}
    \noindent Using this result, we find that a continuous function is measurable- since $(-\infty, a)$ is open for all $a \in \mathbb{R}$, its preimage $f^{-1}(-\infty, a)$ is open, meaning that it is measurable.

    We will now look at some simple results about measurable functions.
    \begin{proposition}
        Let $f, g \colon \mathbb{R} \to \mathbb{R}$ be measurable functions. Then,
        \begin{itemize}
            \item $f + g$ is measurable;
            \item $fg$ is measurable; and
            \item $cf$ is measurable for all $c \in \mathbb{R}$.
        \end{itemize}
    \end{proposition}
    \begin{proof}
        \hspace*{0pt}
        \begin{itemize}
            \item Let $a \in \mathbb{R}$. We claim that
            \[(f + g)^{-1} (-\infty, a) = \bigcup_{r \in \mathbb{Q}} (f^{-1}(-\infty, r) \cap g^{-1}(-\infty, a - r)).\]
            First, let $x \in (f + g)^{-1} (-\infty, a)$. In that case, $f(x) + g(x) < a$, meaning that $f(x) < a - g(x)$. Since $\mathbb{Q}$ is dense in $\mathbb{R}$, we can find an $r \in \mathbb{Q}$ such that $f(x) < r < a - g(x)$, i.e. $f(x) < r$ and $g(x) < a - r$. That is, $x \in f^{-1}(-\infty, r) \cap g^{-1}(-\infty, a -r)$. Hence,
            \[(f + g)^{-1} (-\infty, a) \subseteq \bigcup_{r \in \mathbb{Q}} (f^{-1}(-\infty, r) \cap g^{-1}(-\infty, a - r)).\]
            Now, let $x \in f^{-1}(-\infty, r) \cap g^{-1}(-\infty, a - r)$. In that case, $f(x) < r$ and $g(x) < a - r$, meaning that $f(x) + g(x) < a$. So, $(f + g)^{-1}(x) \in (-\infty, a)$. This implies that 
            \[(f + g)^{-1} (-\infty, a) \subseteq \bigcup_{r \in \mathbb{Q}} (f^{-1}(-\infty, r) \cap g^{-1}(-\infty, a - r)).\]
            Next, since $f$ and $g$ are measurable, and $\sigma$-algebras are closed under intersections and countable unions, we find that $f + g$ is measurable.

            \item We first show that $f^2$ is measurable. So, let $a \in \mathbb{R}$. If $a < 0$, then $(f^2)^{-1}(a, \infty) = \mathbb{R}$ is measurable. Otherwise,
            \begin{align*}
                (f^2)^{-1}(a, \infty) &= \{x \in \mathbb{R} \mid f(x)^2 > a\} \\
                &= \{x \in \mathbb{R} \mid f(x) < -\sqrt{a} \textrm{ or } f(x) > \sqrt{a}\} \\
                &= f^{-1}(-\infty, -\sqrt{a}) \cup f^{-1}(\sqrt{a}, \infty).
            \end{align*}
            Since $f$ is measurable, we find that $f^2$ is measurable. Now, 
            \[4fg = (f + g)^2 - (f - g)^2,\]
            and since $(f + g)^2$ and $(f - g)^2$ are measurable, we find that $fg$ is measurable.

            \item If $c = 0$, then $cf(x) = 0$ for all $x \in \mathbb{R}$, meaning that
            \[(cf)^{-1}(-\infty, a) = \begin{cases}
                \varnothing & a \leq 0 \\
                \mathbb{R} & \textrm{otherwise}.
            \end{cases}\]
            Hence, $cf$ is measurable. Otherwise, if $c > 0$, then
            \begin{align*}
                (cf)^{-1}(-\infty, a) &= \{x \in \mathbb{R} \mid cf(x) < a\} \\
                &= \{x \in \mathbb{R} \mid f(x) < a/c\} \\
                &= f^{-1}(-\infty, a/c).
            \end{align*}
            Since $f$ is measurable, we find that $f^{-1}(-\infty, a/c)$ is measurable. Hence, $cf$ is measurable. Finally, if $c < 0$, then
            \begin{align*}
                (cf)^{-1}(-\infty, a) &= \{x \in \mathbb{R} \mid cf(x) < a\} \\
                &= \{x \in \mathbb{R} \mid -cf(x) > -a\} \\
                &= (-cf)^{-1}(-a, \infty).
            \end{align*}
            Since $-cf$ is measurable, we conclude that $cf$ is measurable.
        \end{itemize}
    \end{proof}
    \begin{proposition}
        Let $f \colon \mathbb{R} \to \mathbb{R}$ be measurable. Then,
        \begin{itemize}
            \item the function $f^+ = \max(0, f)$ is measurable;
            \item the function $f^- = \max(0, -f)$ is measurable; and
            \item the function $|f|$ is measurable.
        \end{itemize}
    \end{proposition}
    \begin{proof}
        \hspace*{0pt}
        \begin{itemize}
            \item By construction $f^+(x) \geq 0$ for all $x \in \mathbb{R}$. Hence, for $a \in (-\infty, 0)$, $(f^+)^{-1}(-\infty, a) = \varnothing$ is measurable. Now, for $a \in [0, \infty)$, we find that $f^+(x) = a$ if and only if $f(x) \leq a$, meaning that 
            \[(f^+)^{-1}(-\infty, a) = f^{-1}(-\infty, a)\]
            is measurable. Hence, $f^+$ is measurable.

            \item Since $f^- = (-f)^+$, we find that $f^-$ is measurable.
            
            \item Let $a \in \mathbb{R}$. Since $|f(x)| \geq 0$ for all $x \in \mathbb{R}$, we find that $f^{-1}(a, \infty) = \mathbb{R}$ if $a \leq 0$. Now, if $a > 0$, then
            \begin{align*}
                |f|^{-1}(a, \infty) &= \{x \in \mathbb{R} \mid |f(x)| > a\} \\
                &= \{x \in \mathbb{R} \mid f(x) < -a \textrm{ or } f(x) > a\} \\
                &= f^{-1}(-\infty, -a) \cup f^{-1}(a, \infty).
            \end{align*}
            Since $f$ is measurable, both $f^{-1}(-\infty, a)$ and $f^{-1}(a, \infty)$ are measurable. Hence, $|f|^{-1}(a, \infty)$ is measurable.
        \end{itemize}
    \end{proof}

    We now define $\inf$ and $\sup$ for a sequence of functions $(f_n)_{n=1}^\infty$, $f_n \colon \mathbb{R} \to \mathbb{R}$:
    \begin{align*}
        (\inf f_n)(x) &= \inf_{n=1}^\infty f_n(x) & (\sup f_n)(x) &= \sup_{n=1}^\infty f_n(x).
    \end{align*}
    It turns out that if $(f_n)$ is a sequence of measurable functions, then $\inf f_n$ and $\sup f_n$ are measurable. In particular, the functions
    \begin{align*}
        (\limsup f_n)(x) &= \sup_{n=1}^\infty \inf_{k=n}^\infty f_k(x) & (\liminf f_n)(x) &= \inf_{n=1}^\infty \sup_{k=n}^\infty f_k(x)
    \end{align*}
    are measurable.
    \newpage

    \section{Simple Functions and Integration}
    In this section, we will cover simple functions and define integration for simple functions. Since simple functions can be used to approximate any measurable function, we will use this to define integration for measurable functions later.

    \begin{definition}
        Let $E \subseteq \mathbb{R}$. The \emph{characteristic function} $\chi_E \colon \mathbb{R} \to \mathbb{R}$ is given by
        \[\chi_E(x) = \begin{cases}
            1 & x \in E \\
            0 & x \not\in E.
        \end{cases}\]
    \end{definition}
    \noindent For $a \in \mathbb{R}$, we have
    \[(\chi_E)^{-1}(a, \infty) = \begin{cases}
        \mathbb{R} & a < 0 \\
        E & 0 \leq a < 1 \\
        \varnothing & \textrm{otherwise}.
    \end{cases}\]
    Hence, $\chi_E$ is measurable if and only if $E$ is measurable. 
    
    We now define simple functions.
    \begin{definition}
        Let $f \colon \mathbb{R} \to \mathbb{R}$ be a function. We say that $f$ is \emph{simple} if there exists a sequence of pairwise disjoint sets $(E_j)_{j=1}^n$ and $c_j \in \mathbb{R}$ such that
        \[f(x) = \sum_{j=1}^n c_j \chi_{E_j}(x)\]
        for all $x \in \mathbb{R}$. 
    \end{definition}
    \noindent It turns out that a simple function is measurable if and only if each $E_j$ is measurable.
    \begin{proposition}
        Let $f \colon \mathbb{R} \to \mathbb{R}$ be a simple function, with
        \[f(x) = \sum_{j=1}^n c_j \chi_{E_j}(x)\]
        for all $x \in \mathbb{R}$, with $c_i = c_j$ if and only if $i = j$. Then, $f$ is measurable if and only if $E_j$ is measurable for all $1 \leq j \leq n$.
    \end{proposition}
    \begin{proof}
        Assume that $E_j$ is measurable for all $1 \leq j \leq n$. In that case, we know that $\chi_{E_j}$ is measurable for all $1 \leq j \leq n$. Since scalar products and finite sums of measurable functions is measurable, we conclude that $f$ is measurable.

        Now, assume that $E_j$ isn't measurable for some $1 \leq j \leq n$. In that case, we know that
        \[f^{-1}(c_j) = E_j\]
        isn't measurable. We have
        \[f^{-1}(-\infty, c_j] \setminus f^{-1}(-\infty, c_j) = f^{-1}(c_j),\]
        so if $f$ was measurable, then $f^{-1}(c_j)$ would also be measurable. Hence, $f$ cannot be measurable.
    \end{proof}
    
    Also, every (measurable) function $f \colon \mathbb{R} \to \mathbb{R}$ can be approximated pointwise by a sequence of simple (measurable) functions.
    \begin{proposition}
        Let $f \colon \mathbb{R} \to \mathbb{R}$ be a function. Then, there exists a sequence of simple measurable functions $(s_k)_{k=1}^\infty$ such that $s_k \to f$ pointwise.
    \end{proposition}
    \begin{proof}
        Without loss of generality, assume that $f \geq 0$, i.e. $f(x) \geq 0$ for all $x \in \mathbb{R}$. For $k \in \mathbb{Z}_{\geq 1}$, define the set
        \[E_j^k = f^{-1} [\tfrac{j-1}{2^k}, \tfrac{j}{2^k}]\]
        for $1 \leq k \leq 2^{2k}$, and let $E = f^{-1}(2^k, \infty)$. Then, define the function
        \[s_k = \sum_{j=1}^{2^{2k}} \frac{j-1}{2^k} \chi_{E_j^k} + 2^k f^{-1} (E).\]
        Since the intervals are getting finer and spanning a larger interval (without bound), we find that $s_k \to f$ pointwise. Moreover, since open sets are Borel, we find that $E_j$ and $E$ are measurable for $1 \leq j \leq 2^{2k}$. So, $s_k$ are simple measurable for $k \in \mathbb{Z}_{\geq 1}$.
    \end{proof}
    % TODO: Illustrate with an example

    We now define integration.
    \begin{definition}
        Let $s$ be a simple function and $A$ a measurable function, with
        \[s = \sum_{i=1}^n c_i \chi_{E_i},\]
        for $1 \leq i \leq j$. We define the \emph{integral of $s$ over $A$} by
        \[\int_A s \ d\mu := \sum_{i=1}^n c_i \cdot \mu(A \cap E_i).\]
        For a measurable function $f \geq 0$, we define the \emph{integral}
        \[\int_A f \ d\mu := \sup \left\{\int_A s \ d\mu \mid 0 \leq s \leq f \textrm{ simple measurable}\right\}.\]
        If the value is finite, we say that $f$ is \emph{integrable}. For a measurable function $f$, we say that $f$ is \emph{integrable} if $f^+ = \max(0, f)$ and $f^- = \max(0, -f)$ are both integrable, with integral
        \[\int_A f \ d\mu := \int_A f^+ \ d\mu - \int_A f^- \ d\mu.\]
        The set $L^1(\mathbb{R}, \mu) = L^1$ the collection of \emph{Lebesgue-integrable} functions.
    \end{definition}

    We will now show some properties about integration.
    \begin{proposition}
        Let $A \subseteq \mathbb{R}$. Then, the integral
        \[\int_{A} f \ d\mu = 0\]
        for every $f \in L^1$ if and only if $\mu(A) = 0$.
    \end{proposition}
    \begin{proof}
        First, assume that $\mu(A) = 0$. In that case, for $0 \leq s \leq f$ simple measurable, with
        \[s = \sum_{i=1}^{n} c_i \chi_{E_i},\]
        we know that $\mu(A \cap E_i) \leq \mu(A) = 0$ for all $0 \leq i \leq n$, and so
        \[\int_A s \ d\mu = \sum_{i=1}^{n} c_i \cdot \mu(A \cap E_i) = 0.\]
        Hence,
        \[\int_A f \ d\mu = \sup \{0\} = 0.\]

        Now, assume that the integral
        \[\int_A f \ d\mu = 0\]
        for every $f \in L^1$. In that case, let $f$ be the constant 1 function. Since this is simple measurable, we find that
        \[0 = \int_A f \ d\mu = 1 \cdot \mu(A \cap \mathbb{R}) = \mu(A).\]
    \end{proof}
    \begin{proposition}
        Let $f, g \in L^1$.
        \begin{enumerate}
            \item Then, $f + g \in L^1$, with
            \[\int_{\mathbb{R}} f + g \ d\mu = \int_{\mathbb{R}} f \ d\mu + \int_{\mathbb{R}} g \ d\mu.\]
            
            \item If $f \leq g$, then 
            \[\int_{\mathbb{R}} f \ d\mu \leq \int_{\mathbb{R}} g \ d\mu.\]
            
            \item The function $|f| \in L^1$, with 
            \[\left|\int_{\mathbb{R}} f \ d\mu\right| \leq \int_{\mathbb{R}} |f| \ d\mu.\]
            
            \item If 
            \[\mu(\{x \in \mathbb{R} \mid f(x) \neq g(x)\}) = 0,\]
            then 
            \[\int_{\mathbb{R}} f \ d\mu = \int_{\mathbb{R}} g \ d\mu.\]
        \end{enumerate}
    \end{proposition}
    \begin{proof}
        We will not prove (1) yet.
        \begin{enumerate}
            \addtocounter{enumi}{1}
            \item First, assume that $0 \leq f \leq g$. In that case, for $0 \leq s \leq f$ simple measurable, we also have $0 \leq s \leq g$. Hence, by the supremum property, we find that
            \[\int_{\mathbb{R}} f \ d\mu \leq \int_{\mathbb{R}} g \ d\mu.\]
            Now, if $f \leq g$, then we have $f^+ \leq g^+$ and $f^- \geq g^-$, meaning that
            \[\int_{\mathbb{R}} f \ d\mu = \int_{\mathbb{R}} f^+ \ d\mu - \int_{\mathbb{R}} f^- \ d\mu \leq \int_{\mathbb{R}} g^+ \ d\mu - \int_{\mathbb{R}} g^- = \int_{\mathbb{R}} g \ d\mu.\]
            
            \item By definition, if $f \in L^1$, then $f^+, f^- \in L^1$, meaning that $|f| = f^+ + f^- \in L^1$. Moreover,
            \begin{align*}
                \left|\int_{\mathbb{R}} f \ d\mu\right| &= \left|\int_{\mathbb{R}} f^+ \ d\mu - \int_{\mathbb{R}} f^- \ d\mu \right| \\
                &\leq \int_{\mathbb{R}} f^+ \ d\mu + \int_{\mathbb{R}} f^- \ d\mu \\
                &= \int_{\mathbb{R}} |f| \ d\mu.
            \end{align*}
            
            \item Let
            \[E = \mu(\{x \in \mathbb{R} \mid f(x) \neq g(x)\}).\]
            In that case, $(f - g)^{-1}(\mathbb{R} \setminus \{0\}) = E$. Hence, for $0 \leq s \leq |f - g|$, we also have $s^{-1}(\mathbb{R} \setminus \{0\}) \subseteq E$, i.e. $\mu(s^{-1}(\mathbb{R} \setminus \{0\})) = 0$. This implies that for $t \in \mathbb{R} \setminus \{0\}$, $\mu(s^{-1}(t)) = 0$. So,
            \[\int_{\mathbb{R}} s \ d\mu = \sum_{t \in s(\mathbb{R})} t \cdot s^{-1}(t) = \sum_{t \in s(\mathbb{R} \setminus \{0\})} t \cdot s^{-1}(t) = \sum_{t \in s(\mathbb{R} \setminus \{0\})} 0 = 0.\]
            Hence,
            \[\int_{\mathbb{R}} |f - g| \ d\mu = \sup \{0\} = 0.\]
            This implies that
            \[0 \leq \left|\int_{\mathbb{R}} f - g \ d\mu \right| \leq \int_{\mathbb{R}} |f - g| \ d\mu = 0,\]
            meaning that
            \[\int_{\mathbb{R}} f - g \ d\mu = 0.\]
            So,
            \[\int_{\mathbb{R}} f \ d\mu = \int_{\mathbb{R}} g \ d\mu.\]
        \end{enumerate}
    \end{proof}
    \newpage

    \section{Convergence Theorems}
    In this section, we will cover some convergence theorems. They tell us when, given a sequence of measurable functions $(f_k)_{k=1}^\infty$, do we have
    \[\int_{\mathbb{R}}\lim_{k \to \infty} f_k \ d\mu = \lim_{k \to \infty} \int_{\mathbb{R}} f_k \ d\mu?\]
    It turns out that Lebesgue integration is more flexible than Riemann integration- the equality above holds in more cases for Lebesgue integration than in Riemann integration. To see this, enumerate $\mathbb{Q} = \{r_1, r_2, \dots\}$. Define the functions $(f_k)_{k=1}^\infty$ by $f_k = \chi_{E_k}$, where 
    \[E_k = \{r_1, r_2, \dots, r_k\}.\]
    Then, 
    \[\int_{-\infty}^\infty f_k(x) \ dx = 0\]
    since $E_k$ is finite. Now, we have $f_k \to f$ pointwise, where $f = \chi_{\mathbb{Q}}$. However, the integral
    \[\int_{-\infty}^{\infty} f(x) \ dx \]
    does not exist. On the other hand, since $\mu(\mathbb{Q}) = 0$, we find that
    \[\int_{\mathbb{R}} f \ d\mu = 0,\]
    so the limit of the integrals is the integral of the limit.

    We will now consider the convergence theorems, starting with the Monotone Convergence Theorem.
    \begin{theorem}[Monotone Convergence Theorem]
        Let $(f_k)_{k=1}^\infty$ be an increasing sequence of non-negative measurable functions, i.e. $f_{k+1} \geq f_k$ for all $k \in \mathbb{Z}_{\geq 1}$ (almost everywhere). Then, the limit function $f = \sup_{k=1}^\infty f_k$ is non-negative and measurable, with
        \[\int_{\mathbb{R}} f \ d\mu = \lim_{k \to \infty} \int_{\mathbb{R}} f_k \ d\mu.\]
    \end{theorem}
    \begin{proof}
        We know that the supremum of a sequence of measurable functions is measurable, so $f$ is measurable. Moreover, the supremum of non-negative functions is still non-negative. We know that $f_k \leq f$ for all $k \in \mathbb{Z}_{\geq 1}$ (almost everywhere). Hence,
        \[\int_{\mathbb{R}} f_k \ d\mu \leq \int_{\mathbb{R}} f \ d\mu.\]
        So,
        \[\lim_{k \to \infty} \int_{\mathbb{R}} f_k \ d\mu \leq \int_{\mathbb{R}} f \ d\mu.\]

        Now, let $\alpha \in (0, 1)$ and a simple measurable function $0 \leq s \leq f$. For $k \in \mathbb{Z}_{\geq 1}$, define the set
        \[E_k = \{t \in \mathbb{R} \mid f_k(t) \geq \alpha s(t)\}.\]
        Since $f_k$ and $\alpha s$ are measurable functions, $E_k$ is measurable. Since $f_k$ is an increasing sequence of functions, we have $E_k \supseteq E_{k+1}$ for all $k \in \mathbb{Z}_{\geq 1}$. Moreover, we claim that
        \[\bigcup_{k=1}^\infty E_k = \mathbb{R}.\]
        So, let $t \in \mathbb{R}$. Since $f_k(t) \to f(t)$, we can find a $k \in \mathbb{Z}_{\geq 1}$ such that 
        \[\alpha s(t) \leq \alpha f(t) < f_k(t) \leq f(t),\]
        meaning that $f_k(t) \geq \alpha s(t)$. This implies that
        \[\int_{\mathbb{R}} f_k \ d\mu \geq \int_{E_k} f_k \ d\mu \geq \alpha \int_{E_k} s \ d\mu\]
        for every $k \in \mathbb{Z}_{\geq 1}$. So, as $k \to \infty$, we find that
        \[\int_{\mathbb{R}} f_k \ d\mu \geq \alpha \int_{\mathbb{R}} s \ d\mu.\]
        Since this holds for every $\alpha \in (0, 1)$, we conclude that
        \[\int_{\mathbb{R}} f_k \ d\mu \geq \int_{\mathbb{R}} s \ d\mu\]
        for all $k \in \mathbb{Z}_{\geq 1}$, meaning that
        \[\lim_{k \to \infty} \int_{\mathbb{R}} f_k \ d\mu \geq \int_{\mathbb{R}} s \ d\mu.\]
        Hence,
        \[\int_{\mathbb{R}} f \ d\mu = \sup \left\{\int_{\mathbb{R}} s \ d\mu \mid 0 \leq s \leq f \textrm{ simple measurable}\right\} \leq \lim_{k \to \infty} \int_{\mathbb{R}} f_k \ d\mu.\]
    \end{proof}
    \noindent We can use this result to show that integral is linear.
    \begin{corollary}
        Let $f, g \colon \mathbb{R} \to \mathbb{R}$ be integrable functions. Then, $f + g$ is integrable, with
        \[\int_{\mathbb{R}} f + g \ d\mu = \int_{\mathbb{R}} f \ d\mu + \int_{\mathbb{R}} g \ d\mu.\]
    \end{corollary}
    \begin{proof}
        Without loss of generality, assume that $f, g \geq 0$. We can find an increasing sequence of simple measurable functions $(s_k)_{k=1}^\infty$ and $(t_k)_{k=1}^\infty$ such that $s_k \to f$ and $t_k \to g$ pointwise. By Monotone Convergence Theorem, we know that
        \[\int_{\mathbb{R}} f \ d\mu = \lim_{k \to \infty} \int_{\mathbb{R}} s_k \ d\mu, \qquad \int_{\mathbb{R}} g \ d\mu = \lim_{k \to \infty} \int_{\mathbb{R}} t_k \ d\mu.\]
        Now, for $k \in \mathbb{Z}_{\geq 1}$, we know that
        \[\int_{\mathbb{R}} s_k + t_k \ d\mu = \int_{\mathbb{R}} s_k \ d\mu + \int_{\mathbb{R}} t_k \ d\mu.\]
        Hence, applying Monotone Convergence Theorem yet again, we find that
        \begin{align*}
            \int_{\mathbb{R}} f + g \ d\mu &= \lim_{k \to \infty} \int_{\mathbb{R}} s_k + t_k \ d\mu \\
            &= \lim_{k \to \infty} \int_{\mathbb{R}} s_k \ d\mu + \lim_{k \to \infty} \int_{\mathbb{R}} t_k \ d\mu \\
            &= \int_{\mathbb{R}} f \ d\mu + \int_{\mathbb{R}} g \ d\mu.
        \end{align*}
    \end{proof}
    \noindent We can also extend the Monotone Convergence Theorem to series.
    \begin{corollary}
        Let $(f_k)_{k=1}^\infty$ be a sequence of positive measurable functions. Then, the series $\sum_{k=1}^{\infty} f_k$ is measurable, with
        \[\int_{\mathbb{R}} \sum_{k=1}^{\infty} f_k \ d\mu = \sum_{k=1}^{\infty} \int_{\mathbb{R}} f_k \ d\mu.\]
    \end{corollary}
    \begin{proof}
        The sequence of partial sums is an increasing sequence of positive measurable functions, so applying the Monotone Convergence Theorem gives us the result.
    \end{proof}

    We now prove Fatou's Lemma.
    \begin{lemma}[Fatou's Lemma]
        Let $(f_k)_{k=1}^\infty$ be a sequence of non-negative measurable functions. Then,
        \[\int_{\mathbb{R}} \liminf_{k \to \infty} f_k \ d\mu \leq \liminf_{k \to \infty} \int_{\mathbb{R}} f_k \ d\mu.\]
    \end{lemma}
    \begin{proof}
        Since $(\inf_{n=k}^\infty f_n)_{k=1}^\infty$ is an increasing sequence of non-negative measurable functions, the Monotone Convergence Theorem tells us that 
        \[\int_{\mathbb{R}} \liminf_{k \to \infty} f_k \ d\mu = \lim_{k \to \infty} \int_{\mathbb{R}} \inf_{n=k}^\infty f_n \ d\mu.\]
        By the infimum property, we know that for all $k \in \mathbb{Z}_{\geq 1}$, $\inf_{n=k}^\infty f_n \leq f_n$ for all $n \geq k$. Hence,
        \[\int_{\mathbb{R}} \inf_{n=k}^\infty f_n \ d\mu \leq \int_{\mathbb{R}} f_n \ d\mu\]
        for all $n \geq k$, meaning that
        \[\int_{\mathbb{R}} \inf_{n=k}^\infty f_n \ d\mu \leq \inf \int_{\mathbb{R}} f_n \ d\mu\]
        by the lower bound property. So, we find that
        \[\lim_{k \to \infty} \int_{\mathbb{R}} \inf_{n=k}^\infty f_n \ d\mu \leq \liminf_{k \to \infty} \int_{\mathbb{R}} f_k \ d\mu.\]
        Hence,
        \[\int_{\mathbb{R}} \liminf_{k \to \infty} f_k \ d\mu \leq \liminf_{k \to \infty} \int_{\mathbb{R}} f_k \ d\mu.\]
    \end{proof}
    \noindent This result can be extended to limit inferior of $f_k$.
    \begin{corollary}
        Let $(f_k)_{k=1}^\infty$ be a sequence of non-negative measurable functions, with $f_k \to f$ pointwise. Then,
        \[\int_{\mathbb{R}} f \ d\mu \leq \liminf_{k \to \infty} \int_{\mathbb{R}} f_k \ d\mu.\]
    \end{corollary}
    \begin{proof}
        Since $f_k \to f$, we find that
        \[\liminf_{k \to \infty} f_k = f.\]
        Hence,
        \[\int_{\mathbb{R}} f \ d\mu \leq \liminf_{k \to \infty} \int_{\mathbb{R}} f_k \ d\mu.\]
    \end{proof}

    We now show the Dominated Convergence Theorem.
    \begin{theorem}[Dominated Convergence Theorem]
        Let $(f_k)_{k=1}^\infty$ be a sequence of integrable functions and let $f$ be a function with $f_k \to f$ pointwise almost everywhere. If there exists an integrable function $g$ such that $|f_k| \leq g$ for all $k \in \mathbb{Z}_{\geq 1}$, then $f$ is integrable, with
        \[\int_{\mathbb{R}} f \ d\mu = \lim_{k \to \infty} \int_{\mathbb{R}} f_k \ d\mu.\]
    \end{theorem}
    \begin{proof}
        Since $|f_k| \leq g$ for all $k \in \mathbb{Z}_{\geq 1}$, we find that
        \[|f| = \lim_{k \to \infty} |f_k| \leq g.\]
        Hence,
        \[\left|\int_{\mathbb{R}} f \ d\mu\right| \leq \int_{\mathbb{R}} |f| \ d\mu \leq \int_{\mathbb{R}} g \ d\mu.\]
        Since $g$ is integrable, this implies that $f$ is integrable.

        Now, let $k \in \mathbb{Z}_{\geq 1}$. Since $|f_k| \leq g$, we know that $g + f_k \geq 0$ and $g - f_k \geq 0$. Hence, Fatou's Lemma tells us that
        \begin{align*}
            \int_{\mathbb{R}} g \ d\mu + \int_{\mathbb{R}} f \ d\mu = \int_{\mathbb{R}} \liminf_{k \to \infty} g + f_k \ d\mu &\leq \int_{\mathbb{R}} g \ d\mu + \liminf_{k \to \infty} \int_{\mathbb{R}} f_k \ d\mu \\
            \int_{\mathbb{R}} g \ d\mu - \int_{\mathbb{R}} f \ d\mu = \int_{\mathbb{R}} \liminf_{k \to \infty} g - f_k \ d\mu &\leq \int_{\mathbb{R}} g \ d\mu - \limsup_{k \to \infty} \int_{\mathbb{R}} f_k \ d\mu.
        \end{align*}
        This implies that
        \[\limsup_{k \to \infty} \int_{\mathbb{R}} f_k \ d\mu \leq \int_{\mathbb{R}} f \ d\mu \leq \liminf_{k \to \infty} \int_{\mathbb{R}} f_k \ d\mu.\]
        Hence,
        \[\lim_{k \to \infty} \int_{\mathbb{R}} f_k \ d\mu = \int_{\mathbb{R}} f \ d\mu.\]
    \end{proof}

    Using the Dominated Convergence Theorem, we can show that Riemann and Lebesgue integration are equal whenever possible.
    \begin{proposition}
        Let $f \colon [a, b] \to \mathbb{R}$ be Riemann integrable. Then, $f$ is Lebesgue integrable, with 
        \[ \int_{a}^{b} f(t) \ dt = \int_{[a, b]} f \ d\mu.\]
    \end{proposition}
    \begin{proof}
        Without loss of generality, assume that $f \geq 0$. We can find a sequence of simple measurable functions $(s_k)_{k=1}^\infty$ such that $0 \leq s_k \leq f$ and $s_k \to f$. Since $f$ is Riemann integrable, it is bounded. So, there exists an $L > 0$ such that $0 \leq f(t) < L$ for all $t \in [a, b]$. In particular, $0 \leq s_k < L$ for all $k \in \mathbb{Z}_{\geq 1}$. Since the constant function $L$ is Lebesgue integrable, we can apply Dominated Convergence Theorem to conclude that $f$ is measurable, with
        \[\lim_{k \to \infty} \int_{[a, b]} s_k \ d\mu = \int_{[a, b]} f \ d\mu.\]
        Since $s_k$ is simple, we know that its integral is equal a finite sum, and so
        \[\lim_{k \to \infty} \int_{[a, b]} s_k \ d\mu = \int_{a}^{b} f(t) \ dt.\]
        Hence,
        \[ \int_{a}^{b} f(t) \ dt = \int_{[a, b]} f \ d\mu.\]
    \end{proof}
    \noindent We can extend this result to infinite integrals- if $f$ is Riemann integrable on $[a, b]$ for all $b > a$, then $f$ is Lebesgue integrable on $[a, \infty)$, with
    \[\int_{a}^{\infty} f(t) \ dt = \int_{[a, \infty)} f  d\mu.\]
    \newpage

    \section{Function Spaces}
    In this section, we define function spaces. We shall connect sequence and function spaces.

    \begin{definition}
        The function space $L^1(X)$ is the set of integrable functions $X \to \mathbb{R}$.
    \end{definition}
    \noindent Since $L^1(X)$ is closed under addition and scalar multiplication, it is a vector space. Moreover, it is a normed vector space, with
    \[\lVert f \rVert_1 = \int_{X} |f| \ d\mu.\]
    However, this is not a norm on the set of integrable functions- we know that if $f$ is zero almost everywhere, then $\int_X f = 0$, even if $f \neq 0$. So, to make it a normed vector space, we quotient by functions that are equal almost everywhere. In general, the function space $L^p(X)$ is the set of functions $f \colon X \to \mathbb{R}$ such that $|f|^p$ are integrable, with norm
    \[\lVert f \rVert_p = \left(\int_X |f|^p \ d\mu\right)^{1/p}.\]
    We know that $C[0, 1] \subseteq L^p(X)$ for $p \in [1, \infty)$. Moreover, the space $L^p(X)$ is complete (if $X$ is complete).

    In the set of natural numbers $\mathbb{Z}_{\geq 1}$, we define the counting measure $\tau$ on $\mathcal{P}(\mathbb{Z}_{\geq 1})$ by $\tau(X) = |X|$. It turns out that $L^p(\mathbb{Z}_{\geq 1}) = \ell^p$ for all $p \in [1, \infty)$. To see this, let $f \colon \mathbb{Z}_{\geq 1} \to \mathbb{R}$ be measurable. In that case, define the sequence of functions $(f_k)_{k=1}^\infty$ by 
    \[f_k(n) = \begin{cases}
        f(n) & n = k \\
        0 & \textrm{otherwise}.
    \end{cases}\]
    Since $f_k \geq 0$ are also measurable, we find that
    \[\int_{\mathbb{Z}_{\geq 1}} f \ d\tau = \sum_{k=1}^\infty \int_{\mathbb{Z}_{\geq 1}} f_k \ d\tau.\]
    Since $f_k$ is also simple, we find that
    \[\int_{\mathbb{Z}_{\geq 1}} f_k \ d\tau = f(n) \cdot \tau(\{k\}) = f(k)\]
    for all $k \in \mathbb{Z}_{\geq 1}$. Hence,
    \[\int_{\mathbb{Z}_{\geq 1}} f \ d\tau = \sum_{k=1}^\infty f(k).\]
    So, if we write $f$ as a sequence $(x_k)_{k=1}^\infty$, we find that the integral corresponds to the series. Hence, $L^p(\mathbb{Z}_{\geq 1}) = \ell^p$ for all $p \in [1, \infty)$.

    To show that $L^p$ space is a norm, we need to show that it satisfies the triangle inequality. We do this in many steps.
    \begin{lemma}
        Let $a, b \geq 0$ and $\lambda \in (0, 1)$. Then,
        \[a^\lambda b^{1 - \lambda} \leq \lambda a + (1 - \lambda) b,\]
        with equality if and only if $a = b$.
    \end{lemma}
    \begin{proposition}[Holder's Inequality]
        Let $p \in [1, \infty)$ and let $q > 0$ such that 
        \[\frac{1}{p} + \frac{1}{q} = 1.\]
        Then, for $f, g$ measurable,
        \[\lVert f g \rVert_1 \leq \lVert f \rVert_p \lVert g \rVert_q.\]
        In particular, if $f \in L^p$ and $g \in L^q$, then $fg \in L^1$, with equality if and only if $\alpha |f|^p = \beta |g|^q$ for $\alpha, \beta > 0$.
    \end{proposition}
    \begin{proof}
        If $\lVert f \rVert_p = 0$, then $f$ is zero almost everywhere, meaning that $fg$ is zero almost everywhere. Hence, $\lVert fg \rVert_1 = 0$ as well. The same holds if $\lVert g \rVert_q = 0$. Now, if $\lVert f \rVert_p = \infty$ or $\lVert g \rVert_q = \infty$, then the result holds trivially. Otherwise, we have $\lVert f \rVert_p, \lVert g \rVert_q \in (0, \infty)$. Without loss of generality, we can assume that $\lVert f \rVert_p = \lVert g \rVert_q = 1$ \sidefootnote{We know that $\lVert . \rVert_p$ respects scalar multiplication, so we can rescale as necessary.}. Now, let $x \in \mathbb{R}$. Let $a = |f(x)|^p, b = |g(x)|^q$ and $\lambda = 1/p$. By the lemma, we find that
        \begin{align*}
            a^\lambda b^{1 - \lambda} &\leq \lambda a + (1 - \lambda) b \\
            a^{1/p} b^{1/q} &\leq \frac{a}{p} + \frac{b}{q} \\
            |f(x) g(x)| &\leq \frac{|f(x)|^p}{p} + \frac{|g(x)|^q}{q}.
        \end{align*}
        Hence,
        \begin{align*}
            \lVert f g \rVert_1 &= \int_{\mathbb{R}} |f(x) g(x)| \ d\mu \\
            &\leq \frac{1}{p} \int_{\mathbb{R}} |f(x)|^p \ d\mu + \frac{1}{q} \int_{\mathbb{R}} |g(x)|^q \ d\mu \\
            &= \frac{1}{p} \lVert f \rVert_p^p + \frac{1}{q} \lVert g \rVert_q^q \\
            &= \frac{1}{p} + \frac{1}{q} \\
            &= 1 = \lVert f \rVert_p \lVert g \rVert_q.
        \end{align*}
        Note that we have equality if $|f|^p = |g|^q$.
    \end{proof}
    \begin{proposition}[Minkowski's Inequality]
        Let $p \in [1, \infty)$, and let $f, g \in L^p$. Then,
        \[\lVert f  + g \rVert_p \leq \lVert f \rVert_p + \lVert g \rVert_p.\]
    \end{proposition}
    \begin{proof}
        First, let $p = 1$. By Triangle Inequality on $\mathbb{R}$, we know that $|f + g| \leq |f| + |g|$. Hence,
        \[\lVert f + g \rVert_1 = \int_{\mathbb{R}} |f + g| \ d\mu \leq \int_{\mathbb{R}} |f| \ d\mu + \int_{\mathbb{R}} |g| \ d\mu = \lVert f \rVert_1 + \lVert g \rVert_1.\]
        Now, let $p > 1$. We know that
        \[|f + g|^p \leq (|f| + |g|)|f + g|^{p-1}\]
        by Triangle Inequality on $\mathbb{R}$. Let $q > 1$ such that $1/p + 1/q = 1$. In that case,
        \begin{align*}
            \int_{\mathbb{R}} |f + g|^p \ d\mu &\leq \int_{\mathbb{R}} |f| \cdot |f + g|^{p-1} \ d\mu + \int_{\mathbb{R}} |g| \cdot |f + g|^{p - 1} \ d\mu \\
            &\leq \lVert f \rVert_p \lVert |f + g|^{p-1} \rVert_q + \lVert g \rVert_p \lVert |f + g|^{p-1} \rVert_q \\
            &= (\lVert f \rVert_p + \lVert g \rVert_q) \left(\int_{\mathbb{R}} |f + g|^{(p-1)q} \ d\mu \right)^{1/q} \\
            &= (\lVert f \rVert_p + \lVert g \rVert_q) \left(\int_{\mathbb{R}} |f + g|^p \ d\mu \right)^{1/q}.
        \end{align*}
        Hence,
        \[\lVert f + g \rVert_p = \left(\int_{\mathbb{R}} |f + g|^p \ d\mu \right)^{1/p} = \left(\int_{\mathbb{R}} |f + g|^p \ d\mu\right)^{1 - 1/q} \leq \lVert f \rVert_p + \lVert g \rVert_q.\]
    \end{proof}
    \noindent Hence, $L^p$ is a normed vector space. 

\end{document}
