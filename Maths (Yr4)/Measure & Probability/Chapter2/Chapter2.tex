\documentclass[a4paper, openany]{memoir}

\usepackage[utf8]{inputenc}
\usepackage[T1]{fontenc} 
\usepackage[english]{babel}

\usepackage{fancyhdr}
\usepackage{float}
\usepackage{bm}

\usepackage{amsmath}
\usepackage{amsthm}
\usepackage{amssymb}
\usepackage{enumitem}
\usepackage{multicol}
\usepackage[bookmarksopen=true,bookmarksopenlevel=2]{hyperref}
\usepackage{tikz}
\usepackage{indentfirst}

\pagestyle{fancy}
\fancyhf{}
\fancyhead[LE]{\leftmark}
\fancyhead[RO]{\rightmark}
\fancyhead[RE, LO]{Measure \& Probability}
\fancyfoot[LE, RO]{\thepage}
\fancyfoot[RE, LO]{Pete Gautam}

\renewcommand{\headrulewidth}{1.5pt}

\theoremstyle{definition}
\newtheorem{definition}{Definition}[section]
\newtheorem{example}[definition]{Example}

\theoremstyle{plain}
\newtheorem{theorem}[definition]{Theorem}
\newtheorem{lemma}[definition]{Lemma}
\newtheorem{proposition}[definition]{Proposition}
\newtheorem{corollary}[definition]{Corollary}

\chapterstyle{thatcher}

\setcounter{chapter}{1}

\begin{document}
    \chapter{Integration}
    \section{Measurable Functions}

    \begin{definition}
        Let $(X, \mathcal{A}, \mu)$ be a measure space. A function $f \colon X \to \mathbb{R} \cup \{\pm \infty\}$ is \emph{measurable} if for all $a \in \mathbb{R}$,
        \[f^{-1}(a, \infty] \in \mathcal{A}.\]
    \end{definition}

    \begin{proposition}
        Let $(X, \mathcal{A}, \mu)$ be a measure space and let $f \colon X \to \mathbb{R} \cup \{\pm \infty\}$ be a function. Then, the following are equivalent:
        \begin{enumerate}
            \item $f$ is measurable;
            \item for all $a \in \mathbb{R}$, $f^{-1}[a, \infty] \in \mathcal{A}$;
            \item for all $a \in \mathbb{R}$, $f^{-1}[-\infty, a) \in \mathcal{A}$;
            \item for all $a \in \mathbb{R}$, $f^{-1}[-\infty, a] \in \mathcal{A}$.
        \end{enumerate}
    \end{proposition}
    \begin{proof}
        For $a \in \mathbb{R}$, we know that
        \[f^{-1}[a, \infty] = [f^{-1}[-\infty, a)]^c, \qquad f^{-1}[-\infty, a] = [f^{-1}(a, \infty]]^c.\]
        So, $1 \iff 4$ and $2 \iff 3$. 
        \begin{itemize}
            \item[$1 \implies 2$] We know that
            \[[a, \infty] = \bigcap_{k=1}^\infty (a - \tfrac{1}{k}, \infty].\]
            So,
            \[f^{-1}[a, \infty] = \bigcup_{k=1}^\infty f^{-1} (a - \tfrac{1}{k}, \infty].\]

            \item[$2 \implies 1$] We know that
            \[(a, \infty] = \bigcup_{k=1}^\infty [a + \tfrac{1}{k}, \infty].\]
            So,
            \[f^{-1}(a, \infty] = \bigcup_{k=1}^\infty f^{-1} [a + \tfrac{1}{k}, \infty].\]
        \end{itemize}
    \end{proof}

    \begin{proposition}
        Let $(X, \mathcal{A}, \mu)$ be a measure space and let $f \colon X \to \mathbb{R} \cup \{\pm \infty\}$ be a measurable function. Then, the function $|f| \colon X \to \mathbb{R} \cup \{\pm \infty\}$ is measurable, where $|f|(x) = |f(x)|$.
    \end{proposition}
    \begin{proof}
        Let $a \in \mathbb{R}$. If $a < 0$, then
        \[|f|^{-1}(a, \infty] = X \in \mathcal{A}.\]
        If $a \geq 0$, then
        \[|f|^{-1}(a, \infty) = f^{-1}(a, \infty] \cup f^{-1}[-\infty, -a) \in \mathcal{A}.\]
        Hence, $|f|$ is measurable.
    \end{proof}

    \begin{proposition}
        Let $(X, \mathcal{A}, \mu)$ be a measure space and let $f \colon X \to \mathbb{R}$ be a constant function. Then, $f$ is measurable.
    \end{proposition}
    \begin{proof}
        Let $x = f(0)$. Then, for all $a \in \mathbb{R}$,
        \[f^{-1}(a, \infty] = \begin{cases}
            \varnothing & a \leq x \\
            X & a > x.
        \end{cases}\]
        So, $f^{-1}(a, \infty] \in \mathcal{A}$. This implies that $f$ is measurable.
    \end{proof}

    \begin{proposition}
        Let $(X, \mathcal{A}, \mu)$ be a measure space and let $A \subseteq X$. Define the function $\chi_A \colon X \to \mathbb{R}$ by
        \[\chi_A = \begin{cases}
            1 & x \in A \\
            0 & \textrm{otherwise}.
        \end{cases}\]
        Then, $\chi_A$ is measurable if and only if $A$ is measurable.
    \end{proposition}
    \begin{proof}
        Let $a \in \mathbb{R}$. We have
        \[\chi_A^{-1}(a, \infty] = \begin{cases}
            \varnothing & a \geq 1 \\
            A & a \in [0, 1) \\
            X & a > 0.
        \end{cases}\]
        So, $\chi_A$ is measurable if and only if $A$ is measurable.
    \end{proof}

    \begin{proposition}
        Let $f \colon \mathbb{R} \to \mathbb{R}$ be a continuous function. Then, $f$ is measurable.
    \end{proposition}
    \begin{proof}
        Let $a \in \mathbb{R}$. Since $[-\infty, a)$ is open and $f$ is continuous, we find that
        \[f^{-1}[-\infty, a)\]
        is open. Hence, the set $f^{-1}[-\infty, a)$ can be written as a union of (open) intervals, meaning that $f$ is measurable.
    \end{proof}

    \begin{proposition}
        Let $(X, \mathcal{A}, \mu)$ be a measure space and let $f, g \colon \mathcal{A} \to \mathbb{R} \cup \{\pm \infty\}$ be measurable functions. Then, the following functions are measurable:
        \begin{enumerate}
            \item $f + g$;
            \item $\lambda f$ for $\lambda \in \mathbb{R}$;
            \item $f \cdot g$;
            \item $f \land g = \max(f, g)$;
        \end{enumerate}
    \end{proposition}
    \begin{proof}
        \hspace*{0pt}
        \begin{enumerate}
            \item Let $h = f \land g$. Then, for $a \in \mathbb{R}$ and $x \in \mathbb{R}$,
            \begin{align*}
                x \in h^{-1}(a, \infty] &\iff f(x) + g(x) < a \\
                &\iff f(x) < a - g(x) \\
                &\iff \exists r \in \mathbb{Q} \textrm{ s.t. } f(x) < r < a - g(x) \\
                &\iff \exists r \in \mathbb{Q} \textrm{ s.t. } f(x) < r \textrm{ and } g(x) < a - r \\
                &\iff x \in f^{-1} (r, \infty] \cap g^{-1} (a - r, \infty].
            \end{align*}
            Hence,
            \[h^{-1}(a, \infty] = \bigcup_{r \in \mathbb{Q}} f^{-1} (r, \infty] \cap g^{-1} (a - r, \infty].\]
            Since this is a countable union of sets in $\mathcal{A}$, we find that $h^{-1}(a, \infty] \in \mathcal{A}$. Hence, $h$ is measurable.
            
            \item If $\lambda = 0$, then $\lambda f$ is a constant, meaning that it is measurable. Otherwise, $\lambda \neq 0$. In that case, let $a \in \mathbb{R}$. We find that
            \begin{align*}
                (\lambda f)^{-1} (a, \infty] &= \{x \in X \mid \lambda f(x) > a\} \\
                &= \{x \in X \mid f(x) > \tfrac{a}{\lambda}\} \\
                &= f^{-1} (\tfrac{a}{\lambda}, \infty].
            \end{align*}
            Hence, since $f$ is measurable, we find that $\lambda f$ is measurable.

            \item We first show that $h = f^2$ is measurable. So, let $a \in \mathbb{R}$. If $a < 0$, then $h^{-1}(a, \infty] = X \in \mathcal{A}$. Then,
            \begin{align*}
                h^{-1}(a, \infty] &= \{x \in X \mid h(x) > a\} \\
                &= \{x \in X \mid f(x) > \sqrt{a} \textrm{ or } f(x) < -\sqrt{a}\} \\
                &= f^{-1}(\sqrt{a}, \infty] \cup f^{-1} [-\infty, -\sqrt{a}).
            \end{align*}
            Since $f$ is measurable, we find that $h$ is measurable. Hence,
            \[f \cdot g = \frac{1}{4} [(f + g)^2 - (f - g)^2]\]
            is measurable.
            
            \item Let $h = f \land g$. Then, for $a \in \mathbb{R}$,
            \begin{align*}
                h^{-1}(a, \infty] &= \{x \in X \mid h(x) > a\} \\
                &= \{x \in X \mid f(x) > a \textrm{ or } g(x) > a\} \\
                &= f^{-1}(a, \infty] \cup g^{-1}(a, \infty].
            \end{align*}
            Since $f$ and $g$ are measurable, we find that $h$ is measurable.
        \end{enumerate}
    \end{proof}

    \begin{definition}
        Let $X$ be a set and $(f_n)_{n=1}^\infty$ be a sequence of functions $f_n \colon X \to \mathbb{R} \cup \{\pm \infty\}$. We define the functions $\inf f_n, \sup f_n \colon X \to \mathbb{R} \cup \{\pm \infty\}$
        \[(\inf f_n)(x) = \inf \{f_n(x) \mid n \in \mathbb{Z}_{\geq 1}\}, \qquad (\sup f_n)(x) = \sup \{f_n(x) \mid n \in \mathbb{Z}_{\geq 1}\}.\]
    \end{definition}

    \begin{definition}
        Let $X$ be a set and $(f_n)_{n=1}^\infty$ be a sequence of functions $f_n \colon X \to \mathbb{R} \cup \{\pm \infty\}$. Then, we define the functions $\inf f_n, \sup f_n, \lim f_n \colon X \to \mathbb{R} \cup \{\pm \infty\}$ by
        \begin{align*}
            (\inf f_n) (x) &= \inf \{f_n(x) \mid n \in \mathbb{Z}_{\geq 1}\} \\
            (\sup f_n) (x) &= \sup \{f_n(x) \mid n \in \mathbb{Z}_{\geq 1}\} \\
            (\lim f_n) (x) &= \lim_{n \to \infty} f_n(x). 
        \end{align*}
        The function $\lim f_n$ need not exist in general.
    \end{definition}

    \begin{definition}
        Let $(a_n)_{n=1}^\infty$ be a sequence in $\mathbb{R}$. Define
        \[\liminf a_n = \sup_{n=1}^\infty \inf \{a_m \mid m \geq n\}, \qquad \limsup_{n \to \infty} a_n = \inf_{n=1}^\infty \sup \{a_m \mid m \geq n\}.\]
    \end{definition}

    \begin{proposition}
        Let $(a_n)_{n=1}^\infty$ be a sequence in $\mathbb{R}$. Then,
        \begin{enumerate}
            \item The value $\liminf_{n \to \infty} a_n$ is the smallest accumulation point of $(a_n)$;
            \item The value $\limsup_{n \to \infty} a_n$ is the largest accumulation point of $(a_n)$.
        \end{enumerate}
    \end{proposition}
    \begin{proof}
        \hspace*{0pt}
        \begin{enumerate}
            \item 
            
            \item 
        \end{enumerate}
    \end{proof}
    
    \begin{proposition}
        Let $(a_n)_{n=1}^\infty$ be a sequence in $\mathbb{R}$. Then,
        \[\liminf_{n \to \infty} a_n \leq \limsup_{n \to \infty} a_n,\]
        with
        \[\liminf_{n \to \infty} a_n = \limsup_{n \to \infty} a_n\]
        if and only if
        \[\liminf_{n \to \infty} a_n = \lim_{n \to \infty} a_n = \limsup_{n \to \infty} a_n.\]
    \end{proposition}
    
    \begin{definition}
        Let $X$ be a set and $(f_n)_{n=1}^\infty$ be a sequence of functions $f_n \colon X \to \mathbb{R} \cup \{\pm \infty\}$. We define the functions $\liminf f_n, \limsup f_n \colon X \to \mathbb{R} \cup \{\pm \infty\}$
        \begin{align*}
            (\liminf f_n)(x) &= \liminf f_n(x) = \sup_{n=1}^\infty \inf \{f_n(x) \mid n \in \mathbb{Z}_{\geq 1}\}, \\
            (\limsup f_n)(x) &= \limsup f_n(x) = \inf_{n=1}^\infty \sup \{f_n(x) \mid n \in \mathbb{Z}_{\geq 1}\}.
        \end{align*}
    \end{definition}

    \begin{proposition}
        Let $X$ be a set and $(f_n)_{n=1}^\infty$ be a sequence of measurable functions $f_n \colon X \to \mathbb{R} \cup \{\pm \infty\}$. Then, the following functions are measurable:
        \begin{enumerate}
            \item $\inf f_n$ and $\sup f_n$;
            \item $\liminf f_n$ and $\limsup f_n$;
            \item the limit $\lim f_n$, if it exists.
        \end{enumerate}
    \end{proposition}
    \begin{proof}
        \hspace*{0pt}
        \begin{itemize}
            \item Let $a \in \mathbb{R}$. Then, for $x \in \mathbb{R}$,
            \begin{align*}
                x \in (\inf f_n)^{-1}(a, \infty] &\iff (\inf f_n)(x) > a \\
                &\iff f_n(x) > a \ \forall n \in \mathbb{Z}_{\geq 1} \\
                &\iff x \in \bigcap_{n=1}^\infty f_n^{-1}(a, \infty].
            \end{align*}
            Since $f_n$ is measurable for all $n \in \mathbb{Z}_{\geq 1}$, we find that 
            \[(\inf f_n)^{-1}(a, \infty] = \bigcap_{n=1}^\infty f_n^{-1}(a, \infty]\]
            is measurable. Hence, $\inf f_n$ is measurable. We have
            \[\sup f_n = -\inf (-f_n),\]
            so $\sup f_n$ is measurable as well.

            \item Define the sequence of functions $(g_n)_{n=1}^\infty$, $g_n \colon X \to \mathbb{R}$ by
            \[g_n(x) = \inf \{f_m(x) \mid m \geq n\}.\]
            By the result above, we know that $(g_n)$ is a sequence of measurable functions. We know that
            \[\liminf f_n = \sup g_n,\]
            so $\liminf f_n$ is measurable. Similarly, $\limsup f_n$ is measurable.
            
            \item If $\lim f_n$ exists, then $\lim f_n = \liminf f_n$. So, $\lim f_n$ is measurable.
        \end{itemize}
    \end{proof}

    \begin{definition}
        Let $X$ be a set and $f \colon X \to \mathbb{R}$ be a function. We say that $f$ is \emph{simple} if the image $f(X)$ is finite.
    \end{definition}

    \begin{definition}
        Let $X$ be a set and $(f_n)_{n=1}^\infty$ a sequence of functions $f_n \colon X \to \mathbb{R}$, and let $f \colon X \to \mathbb{R}$ be a function. 
        \begin{itemize}
            \item We say that $(f_n)$ \emph{converges pointwise} to $f$ if for every $\varepsilon > 0$ and $x \in X$, there exists an $N \in \mathbb{Z}_{\geq 1}$ such that for $n \in \mathbb{Z}_{\geq 1}$, if $n \geq N$, then $|f_n(x) - f(x)| < \varepsilon$.
            \item We say that $(f_n)$ \emph{converges uniformly} to $f$ if for every $\varepsilon > 0$, there exists an $N \in \mathbb{Z}_{\geq 1}$ such that for all $x \in X$ and $n \in \mathbb{Z}_{\geq 1}$, if $n \geq N$, then $|f_n(x) - f(x)| < \varepsilon$.
        \end{itemize}
    \end{definition}

    \begin{theorem}
        Let $(X, \mathcal{A}, \mu)$ be a measure space and $f \colon X \to \mathbb{R}$ be a function. Then, there exists a sequence $(s_n)_{n=1}^\infty$ of simple functions $s_n \colon X \to \mathbb{R}$ such that $s_n \to f$ pointwise. Moreover,
        \begin{enumerate}
            \item if $f$ is bounded, then the sequence $(s_n)$ can be chosen such that $s_n \to f$ uniformly.
            \item if $f \geq 0$, then the sequence $(s_n)$ can be chosen to be positive increasing (i.e. $0 \leq s_1 \leq s_2 \leq \dots$).
            \item if $f$ is measurable, then the sequence $(s_n)$ can be chosen such that $s_n$ is measurable  for all $n \in \mathbb{Z}_{\geq 1}$.
        \end{enumerate}
    \end{theorem}
    \begin{proof}
        \hspace*{0pt}
        \begin{enumerate}
            \setcounter{enumi}{-1}
            % \item Define the function $s_n \colon X \to \mathbb{R}$ by 
            % \[s_n(x) = \begin{cases}
            %     f(y) & x \in [y, y + \tfrac{1}{n}) \\
            %     0 & \textrm{otherwise},
            % \end{cases}\]
            % where $y \in \{-n, -n + \tfrac{1}{n}, -n + \tfrac{2}{n}, \dots, n - \tfrac{1}{n}\}$.
            \item 
            
            \item 
            
            \item 
            
            \item 
        \end{enumerate}
    \end{proof}

    \begin{definition}
        Let $(X, \mathcal{A}, \mu)$ be a measure space, and let $s \colon X \to \mathbb{R}$ be a simple measurable function denoted by
        \[s = \lambda_1 \chi_{E_1} + \lambda_2 \chi_{E_2} + \dots + \lambda_n \chi_{E_n},\]
        where $E_k \in \mathcal{A}$ for $1 \leq i \leq n$. We define the \emph{integral of $s$ over $X$} as
        \[\int_X s \ d\mu = \sum_{k=1}^n \lambda_k \mu(E_k).\]
        In general, for $A \in \mathcal{A}$, we define the \emph{of $s$ over $A$} to be
        \[\int_X s \ d\mu = \sum_{k=1}^n \lambda_k \mu(E_k \cap A).\]
    \end{definition}

    \begin{proposition}
        Let $(X, \mathcal{A}, \mu)$ be a measure space, $s, t \colon X \to \mathbb{R}$ be simple and measurable and let $A \in \mathcal{A}$. Then,
        \begin{enumerate}
            \item $s + t$ is simple and measurable and
            \[\int_A s + t \ d\mu = \int_A s \ d\mu + \int_A t \ d\mu.\]

            \item $s \geq t$ implies
            \[\int_A s \ d\mu \geq \int_A t \ d\mu.\]

            \item for all $\lambda \in \mathbb{R}$, $\lambda s$ is simple and measurable, with
            \[\int_A \lambda s \ d\mu = \lambda \int_A s \ d\mu.\]
        \end{enumerate}
    \end{proposition}
    \begin{proof}
        \hspace*{0pt}
        \begin{enumerate}
            \item 
            \item 
            \item 
        \end{enumerate}
    \end{proof}

    \begin{proposition}
        Let $(X, \mathcal{A}, \mu)$ be a measure space, $s \colon X \to \mathbb{R}$ be a simple measurable function. Then, the function $\nu \colon \mathcal{A} \to \mathbb{R}$ given by
        \[\nu(A) = \int_A s \ d\mu\]
        is a measure.
    \end{proposition}
    \begin{proof}
        Let
        \[s = \lambda_1 \chi_{E_1} + \lambda_2 \chi_{E_2} + \dots + \lambda_n \chi_{E_n}.\]
        We find that
        \begin{align*}
            \nu (\varnothing) &= \int_{\varnothing} s \ d\mu \\
            &= \sum_{k=1}^n \lambda_k \mu (E_k \cap \varnothing) \\
            &= \sum_{k=1}^n \lambda_k \mu (\varnothing) = 0.
        \end{align*}
        Now, let $(A_n)_{n=1}^\infty$ be a sequence of disjoint sets in $\mathcal{A}$. Denote
        \[A = \bigcup_{j=1}^\infty A_j.\]
        We find that
        \begin{align*}
            \nu (A) &= \int_A s \ d\mu \\
            &= \sum_{k=1}^n \lambda_k \mu(E_k \cap A) \\
            &= \sum_{k=1}^n \lambda_k \mu \left(\bigcup_{j=1}^\infty (E_k \cap A_j)\right) \\
            &= \sum_{k=1}^n \lambda_k \cdot \sum_{j=1}^\infty \mu(E_k \cap A_j) \\
            &= \sum_{j=1}^\infty \sum_{k=1}^n \lambda_k \mu(E_k \cap A_j) \\
            &= \sum_{j=1}^\infty \int_{A_j} s \ d\mu = \sum_{j=1}^\infty \nu(A_j).
        \end{align*}
        Hence, $\nu$ is a measure.
    \end{proof}

    \begin{definition}
        Let $(X, \mathcal{A}, \mu)$ be a measure space and $f \colon X \to \mathbb{R}$ be measurable with $f \geq 0$. The \emph{integral of $f$ over $A$}, for some $A \in \mathcal{A}$, is given by
        \[\int_A f \ d\mu := \sup \left\{\int_A s \ d\mu \mid 0 \leq s \leq f \text{ simple measurable} \right\}.\]
        For $f \colon X \to \mathbb{R}$, define
        \[f_+ = \max(f(x), 0), \qquad f_- = \max(-f(x), 0).\]
        We say that \emph{$f$ is integrable over $A$}, for some $A \in \mathcal{A}$, if both
        \[\int_A f_+ \ d\mu < \infty, \qquad \int_A f_- \ d\mu < \infty.\]
        In that case, we define the \emph{integral of $f$ over $A$} by
        \[\int_A f \ d\mu = \int_A f_+ \ d\mu - \int_A f_- \ d\mu.\]
        If $f$ is integrable, we denote $f \in \mathcal{L}_1(X, \mu) = \mathcal{L}(X, \mu)$.
    \end{definition}

    \begin{proposition}
        Let $(X, \mathcal{A}, \mu)$ be a measure space and $f \colon X \to \mathbb{R}$ be a simple measurable function. Then,
        \[\int_A f \ d\mu = \sup \left\{\int_A s \ d\mu \mid 0 \leq s \leq f \text{ simple measurable} \right\}.\]
    \end{proposition}
    \begin{proof}
        
    \end{proof}

    \begin{proposition}
        Let $(X, \mathcal{A}, \mu)$ be a measure space, $f, g \in \mathcal{L}(X, \mu)$ and $A, B \in \mathcal{A}$. Then, 
        \begin{enumerate}
            \item \begin{enumerate}
                \item $\int_A \lambda f \ d\mu = \lambda \int_A f \ d\mu$ for $\lambda \in \mathbb{R}$;
                \item $\int_A (f + g) \ d\mu \geq \int_A f \ d\mu + \int_A g \ d\mu$;
            \end{enumerate}
            \item if $f \leq g$, then $\int_A f \ d\mu \leq \int_A g \ d\mu$;
            \item $f \in \mathcal{L}(X, \mu)$ if and only if $|f| \in \mathcal{L}(X, \mu)$, with
            \[\left|\int_A f \ d\mu\right| \leq \int_A |f| \ d\mu;\]
            \item \begin{enumerate}
                \item if $\mu(A) = 0$, then
                \[\int_A f \ d\mu = 0;\]
                \item if $A \subseteq B$ and $\mu(B \setminus A) = 0$,
                \[\int_A f \ d\mu = \int_B f \ d\mu;\]
                \item If 
                \[\mu (\{x \in X \mid f(x) \neq g(x)\}) = 0,\]
                then 
                \[\int_A f \ d\mu = \int_A g \ d\mu.\]
            \end{enumerate}
        \end{enumerate}
    \end{proposition}
    \begin{proof}
        \hspace*{0pt}
        \begin{enumerate}
            \item \begin{enumerate}
                \item First, assume that $f \geq 0$. If $\lambda = 0$, then
                \[\int_A \lambda f \ d\mu = 0 = \lambda \int_A f \ d\mu.\]
                Otherwise, if $\lambda > 0$, then
                \begin{align*}
                    \int_A \lambda f \ d\mu &= \sup \left\{\int_A \lambda s \ d\mu \mid 0 \leq s \leq \lambda f \textrm{ sm}\right\} \\
                    &= \sup \left\{\lambda \int_A \frac{1}{\lambda}s \ d\mu \mid 0 \leq \frac{1}{\lambda}s \leq f \textrm{ sm}\right\} \\
                    &= \lambda \sup \left\{\int_A t \ d\mu \mid 0 \leq t \leq f \textrm{ sm}\right\} \\
                    &= \lambda \int_A f \ d\mu.
                \end{align*}
                Now, for a general $f$, we have
                \begin{align*}
                    \lambda \int_A f \ d\mu &= \lambda \left(\int_A f_+ \ d\mu - \int_A f_- \ d\mu\right) \\
                    &= \lambda \int_A f_+ \ d\mu - \lambda \int_A f_- \ d\mu \\
                    &= \int_A \lambda f_+ \ d\mu - \int_A \lambda f_- \ d\mu \\
                    &= \int_A \lambda f \ d\mu.
                \end{align*}
                
                \item We find that
                \begin{align*}
                    \int_A f \ d\mu + \int_A g \ d\mu &= \sup \left\{\int_A s \ d\mu \mid 0 \leq s \leq f \textrm{ simple measurable}\right\} \\
                    &+ \sup \left\{\int_A t \ d\mu \mid 0 \leq t \leq g \textrm{ simple measurable}\right\} \\
                    &= \sup \left\{\int_A s \ d\mu + \int_A t \ d\mu \mid 0 \leq s \leq f, 0 \leq t \leq g  \textrm{ s.m.}\right\} \\
                    &= \sup \left\{\int_A (s + t) \ d\mu \mid 0 \leq s + t \leq f + g \textrm{ s.m.} \right\} \\
                    &\geq \sup \left\{\int_A s \ d\mu \mid 0 \leq s \leq f + g \textrm{ s.m.}\right\} \\
                    &= \int_A f + g \ d\mu.
                \end{align*}
            \end{enumerate}

            \item For all $0 \leq s \leq f$, we have $0 \leq s \leq g$. Hence, by the supremum property, we find that
            \[\int_A f \ d\mu \leq \int_A g \ d\mu.\]

            \item We have
            \begin{align*}
                \left| \int_A f \ d\mu\right| &= \left|\int_A f_+ \ d\mu - \int_A f_- \ d\mu\right| \\
                &\leq \int_A f_+ d\mu + \int_A f_- \ d\mu = \int_A |f| \ d\mu.
            \end{align*}
            
            \item \begin{enumerate}
                \item Let $0 \leq s \leq f$ be simple and measurable. Then,
                \[\int_A s \ d\mu = \sum_{t \in s(X)} t \cdot \mu(f^{-1}(t) \cap A) = \sum_{t \in s(X)} t \cdot 0 = 0.\]
                Hence,
                \[\int_A f \ d\mu = \sup \{0\} = 0.\]
                
                \item For any $C \in \mathcal{A}$, we have
                \[\mu(C \cap B) = \mu(C \cap A) + \mu(C \cap B \setminus A) = \mu(C \cap A)\]
                since $C \cap B \setminus A \subseteq B \setminus A$. Hence, for any simple measurable function $s \geq 0$,
                \begin{align*}
                    \int_B s \ d\mu &= \sum_{t \in s(X)} t \cdot \mu(f^{-1}(t) \cap B) \\
                    &= \sum_{t \in s(X)} t \cdot \mu(f^{-1}(t) \cap A) \\
                    &= \int_A s \ d\mu.
                \end{align*}
                This implies that for a function $f \geq 0$,
                \begin{align*}
                    \int_B f \ d\mu &= \sup \left\{\int_B s \ d\mu \mid 0 \leq s \leq f \textrm{ sm}\right\} \\
                    &= \sup \left\{\int_A s \ d\mu \mid 0 \leq s \leq f \textrm{ sm}\right\} \\
                    &= \int_A f \ d\mu.
                \end{align*}
                Hence, for an arbitrary function $f$,
                \begin{align*}
                    \int_B f \ d\mu &= \int_B f_+ \ d\mu - \int_B f_- \ d\mu \\
                    &= \int_A f _+ \ d\mu - \int_A f_- \ d\mu \\
                    &= \int_A f \ d\mu.
                \end{align*}

                \item Assume first that $f \geq 0$. Let
                \[B = \{x \in X \mid f(x) = g(x)\}.\]
                We know that $\mu(B \setminus A) = 0$. Now, let $0 \leq s \leq f$ be simple measurable. Define the function $s' \colon X \to [0, \infty)$ by
                \[s'(x) = \begin{cases}
                    s(x) & x \in B \\
                    0 & \textrm{otherwise}.
                \end{cases}\]
                We have $s'(X) = s(X) \cup \{0\}$, meaning that $s'$ is simple. Moreover, $s'$ is measurable, since 
                \[(s')^{-1}(a, \infty) = \begin{cases}
                    X & x < 0 \\
                    s^{-1}(a, \infty) \cap B & \textrm{otherwise}.
                \end{cases}\]
                Moreover, $0 \leq s' \leq g$ by construction with
                \begin{align*}
                    \int_B s \ d\mu &= \sum_{t \in s(X)} t \cdot \mu(s^{-1}(t) \cap B) \\
                    &= \sum_{t \in s'(X)} t \cdot \mu((s')^{-1}(t) \cap B) \\
                    &= \int_B s' \ d\mu.
                \end{align*}
                Hence,
                \[\left\{\int_B s \ d\mu \mid 0 \leq s \leq f \textrm{ sm}\right\} = \left\{\int_B s \ d\mu \mid 0 \leq s \leq g \textrm{ sm}\right\},\]
                meaning that
                \[\int_A f \ d\mu = \int_B f \ d\mu = \int_B g \ d\mu = \int_A g \ d\mu.\]
                Hence, for an arbitrary measurable $f$,
                \begin{align*}
                    \int_A f \ d\mu &= \int_A f_+ \ d\mu - \int_A f_- \ d\mu \\
                    &= \int_A g_+ \ d\mu - \int_A g_- \ d\mu \\
                    &= \int_A g \ d\mu.
                \end{align*}
            \end{enumerate}
        \end{enumerate}
    \end{proof}
    \newpage

    \section{Convergence Theorems}
    \begin{theorem}[Lebesgue's Montone Convergence Theorem]
        Let $(X, \mathcal{A}, \mu)$ be a measure space and let $(f_n)_{n=1}^\infty$ be a sequence of measurable functions $f_n \colon X \to [0, \infty]$ such that $0 \leq f_1 \leq f_2 \leq \dots$. Define the function $f \colon X \to [0, \infty]$ by $f(x) = \lim_{n \to \infty} f_n(x)$. Then, $f$ is measurable, with
        \[\lim_{n \to \infty} \int_A f_n \ d\mu = \int_A f \ d\mu = \int_A \lim_{n \to \infty} f_n \ d\mu.\]
        In particular, if the sequence $(\int_A f_n \ d\mu)$ is bounded above, then $f$ is integrable.
    \end{theorem}
    \begin{proof}
        By monotonicity, we know that $f_n \leq f$. Hence,
        \[\int_A f_n \ d\mu \leq \int_A f \ d\mu,\]
        meaning that
        \[\lim_{n \to \infty} \int_A f_n \ d\mu \leq \int_A f \ d\mu.\]
        We now show that
        \[\lim_{n \to \infty} \int_A f_n \ d\mu \geq \int_A f \ d\mu.\]
        % TODO
    \end{proof}

    \begin{proposition}
        Let $(X, \mathcal{A}, \mu)$ be a measure space and $(f_n)_{n=1}^\infty$ be a sequence of measurable functions $0 \leq f_1 \leq f_2 \leq \dots$ almost everywhere, i.e. for all $i \in \mathbb{Z}_{\geq 1}$,
        \[\mu (\{x \in X \mid f_i(x) > f_{i+1}(x)\}) = 0.\]
        Then, $f$ is measurable, with
        \[\lim_{n \to \infty} \int_A f_n \ d\mu = \int_A f \ d\mu = \int_A \lim_{n \to \infty} f_n \ d\mu.\]
    \end{proposition}
    \begin{proof}
        
    \end{proof}

    \begin{proposition}
        Let $(X, \mathcal{A}, \mu)$ be a measure space and $(g_n)_{n=1}^\infty$ be a sequence of measurable functions that are non-negative almost everywhere. Then,
        \[\sum_{n=1}^\infty \int_A g_n \ d\mu = \int_A \sum_{n=1}^\infty g_n \ d\mu.\]
    \end{proposition}
    \begin{proof}
        Define the sequence $(f_n)_{n=1}^\infty$ by
        \[f_n = \sum_{k=1}^n g_k.\]
        Since $g_n$ are non-negative, we know that $(f_n)$ is non-decreasing. Hence,
        \begin{align*}
            \sum_{n=1}^\infty \int_A g_n \ d\mu &= \lim_{n \to \infty} \sum_{k=1}^n \int_A g_k \ d\mu \\
            &= \lim_{n \to \infty} \int_A \sum_{k=1}^n g_k \ d\mu \\
            &= \int_A \lim_{n \to \infty} \sum_{k=1}^n g_k \ d\mu \\
            &= \int_A \sum_{n=1}^\infty g_n \ d\mu.
        \end{align*}
    \end{proof}

    \begin{proposition}
        Let $(X, \mathcal{A}, \mu)$ be a measure space and let $f \colon X \to [0, \infty]$ be measurable. Then, $f = 0$ almost everywhere if and only if
        \[\int_A f \ d\mu = 0.\]
        for all $A \in \mathcal{A}$.
    \end{proposition}
    \begin{proof}
        If $f = 0$ almost everywhere, then for all $A \in \mathcal{A}$,
        \[\int_A f \ d\mu = \int_A 0 \ d\mu = 0.\]
        Now, assume that $f \neq 0$ almost everywhere. In that case, let $A = f^{-1}(0, \infty]$, with $\mu(A) = \varepsilon > 0$. Now, define $A_n = f^{-1}(\frac{1}{n}, \infty]$. We know that $A_n \in \mathcal{A}$ for all $n \in \mathbb{Z}_{\geq 1}$, since $f$ is measurable, with
        \[A = \bigcup_{n=1}^\infty A_n.\]
        Hence,
        \[\lim_{n \to \infty} \mu(A_n) = \mu(A) = \varepsilon.\]
        This implies that there exists an $n \in \mathbb{Z}_{\geq 1}$ such that $\mu(A_n) > \frac{\varepsilon}{2} > 0$. Now, define the function $g \colon X \to [0, \infty)$ by $g = \frac{1}{n} \chi_{A_n}$. For $x \in X$, if $x \not\in A_n$, then $g(x) = 0 \leq f(x)$, and if $x \in A_n$, then $g(x) = \frac{1}{n} < f(x)$. So, $g \leq f$, meaning that
        \[\int_X f \ d\mu \geq \int_X g \ d\mu = \frac{1}{n} \mu(A_n) > 0.\]
    \end{proof}

    \begin{lemma}[Fatou's Lemma]
        Let $(X, \mathcal{A}, \mu)$ be a measure space and let $(f_n)_{n=1}^\infty$ be a sequence on non-negative measurable functions on $X$. Then,
        \[\int_A \liminf_{n \to \infty} f_n(x) \ d\mu \leq \liminf_{n \to \infty} \left(\int_A f_n \ d\mu\right).\]
    \end{lemma}
    \begin{proof}
        
    \end{proof}

    \begin{theorem}[Lebesgue's Dominated Convergence Theorem]
        Let $(f_n)_{n=1}^\infty$ be a sequence of measurable functions $f_n \colon X \to \mathbb{R}$ such that $f(x) = \lim_{n \to \infty} f_n(x)$ exists for all $x \in X$. If there exists a $g \colon X \to [0, \infty]$ such that $|f_n| \leq g$ for all $n \in \mathbb{Z}_{\geq 1}$, then
        \begin{enumerate}
            \item $f_n$ and $f$ are integrable for all $n \in \mathbb{Z}$;
            \item $\int_X |f_n - f| \ d\mu \to 0$ as $n \to \infty$;
            \item $\lim_{n \to \infty} \int_X f_n \ d\mu = \int_X \lim_{n \to \infty} f_n \ d\mu = \int_X f \ d\mu$.
        \end{enumerate}
    \end{theorem}
    \begin{proof}
        \hspace*{0pt}
        \begin{enumerate}
            \item Since $g$ is integrable with $|f_n| \leq g$ for all $n \in \mathbb{Z}_{\geq 1}$, we find that
            \[\int_X |f_n| \ d\mu \leq \int_X g \ d\mu < \infty.\]
            So, $|f_n|$ is integrable, meaning that $f_n$ is integrable. Moreover, 
            \[\int_X f \ d\mu \leq \int_X g \ d\mu < \infty,\]
            meaning that $f$ is also integrable.

            \item For $n \in \mathbb{Z}_{\geq 1}$, we have
            \[|f_n - f| \leq |f_n| + |f| \leq g + |f| < 2g,\]
            meaning that $h := g + |f|$ is integrable. Moreover,
            \begin{align*}
                \int_X h \ d\mu &= \int_X \lim_{n \to \infty} h - |f_n - f| \ d\mu \\
                &= \int_X \liminf_{n \to \infty} h - |f_n - f| \ d\mu \\
                &\leq \liminf_{n \to \infty} \int_X h - |f_n - f| \ d\mu \\
                &= \int_X h \ d\mu - \limsup_{n \to \infty} \int_X |f_n - f| \ d\mu.
            \end{align*}
            Hence, we find that
            \[0 \leq \liminf \int_X |f_n - f| \ d\mu \leq \limsup \int_X |f_n - f| \ d\mu \leq 0.\]
            This implies that $\int_X |f_n - f| \ d\mu \to 0$ as $n \to \infty$.

            \item We find that for all $n \in \mathbb{Z}_{\geq 1}$,
            \[\left|\int_X f_n \ d\mu - \int_X f \ d\mu\right| \leq \int_X |f_n - f| \ d\mu,\]
            meaning that
            \[\lim_{n \to \infty} \int_X f_n \ d\mu - \lim_{n \to \infty} \int_X f \ d\mu = 0.\]
        \end{enumerate}
    \end{proof}

    % \begin{proposition}
    %     Let $f \colon \mathbb{R}^2 \to \mathbb{R}$ be a continuous function such that $\frac{\partial f}{\partial t} (t, x)$ exists for all $(t, x) \in \mathbb{R}^2$ and is continuous. For every 
    % \end{proposition}
    % \begin{proof}
        
    % \end{proof}
    
    \begin{proposition}
        Let $f \colon [a, b] \to \mathbb{R}$ be Riemann integrable. Then, $f$ is Lebesgue integrable, with
        \[\int_a^b f(x) \ dx = \int_{[a, b]} f \ d\lambda.\]
    \end{proposition}
    \begin{proof}
        Without loss of generality, assume that $a = 0, b = 1$. Define the sequence of partitions $(P_n)_{n=1}^\infty$ by 
        \[P_n = \{0, \tfrac{1}{2^n}, \dots, 1 - \tfrac{1}{2^n}, 1\}.\]
        Define the functions $(g_n)_{n=1}^\infty, (h_n)_{n=1}^\infty$ by
        \[g_n = \sum_{k=1}^{2^n} m_k \chi_{\left(\tfrac{k-1}{2^n}, \tfrac{k}{2^n}\right]}, \qquad h_n = \sum_{k=1}^{2^n} M_k \chi_{\left(\tfrac{k-1}{2^n}, \tfrac{k}{2^n}\right]},\]
        where $m_k = \inf f(-\tfrac{k-1}{2^n}, \tfrac{k}{2^n}]$ and $M_k = \sup f(-\tfrac{k-1}{2^n}, \tfrac{k}{2^n}]$. For $n \in \mathbb{Z}_{\geq 1}$, the functions $g_n$ and $h_n$ are simple measurable, with
        \begin{align*}
            \int_{[0, 1]} g_n \ d\mu &= \sum_{k=1}^{2^n} m_k \mu \left(\tfrac{k-1}{2^n}, \tfrac{k}{2^n}\right] = L(f, P_n) \\
            \int_{[0, 1]} h_n \ d\mu &= \sum_{k=1}^{2^n} M_k \mu \left(\tfrac{k-1}{2^n}, \tfrac{k}{2^n}\right] = U(f, P_n).
        \end{align*}
        Moreover, for all $n \in \mathbb{Z}_{\geq 1}$, $g_n \leq f \leq h_n$. Since $f$ is Riemann integrable, we find that
        \[L(f, P_n) \to \int_0^1 f(x) \ dx, \qquad U(f, P_n) \to \int_0^1 g(x) \ dx.\]
        Since $f$ is Riemann integrable, it is bounded by some $C > 0$. Hence, we find that $|g_n| \leq C$ and $|h_n| \leq C$ for all $n \in \mathbb{Z}_{\geq 1}$. Since the constant function $C$ is measurable, the Dominated Convergence Theorem tells us that 
        \begin{align*}
            \lim_{n \to \infty} \int_{[0, 1]} g_n \ d\mu &= \int_{[0, 1]} \lim_{n \to \infty} g_n \ d\mu = \int_0^1 f(x) \ dx, \\
            \lim_{n \to \infty} \int_{[0, 1]} h_n \ d\mu &= \int_{[0, 1]} \lim_{n \to \infty} h_n \ d\mu = \int_0^1 f(x) \ dx.
        \end{align*}
        Since
        \[\int_{[0, 1]} g \ d\mu = \int_{[0, 1]} h \ d\mu,\]
        we find that $g = h$ almost everywhere. Since $g \leq f \leq h$, this implies that $f$ is measurable, with $g = f = h$ almost everywhere and
        \[\int_{[0, 1]} f \ d\mu = \int_{[0, 1]} g \ d\mu = \int_0^1 f(x) \ dx.\]
    \end{proof}
    \newpage

    \section{$\mathcal{L}^p$ spaces}
    
    \begin{definition}
        Let $V$ be a real vector space. A map $\lVert . \rVert \colon V \to [0, \infty)$ if for $u, v \in V$ and $\lambda \in \mathbb{R}$,
        \begin{itemize}
            \item $\lVert u + v \rVert \leq \lVert u \rVert + \lVert v \rVert$;
            \item $\lVert \lambda v \rVert = |\lambda| \lVert v \rVert$;
            \item $\lVert v \rVert = 0$ if and only if $v = 0$.
        \end{itemize}
    \end{definition}

    \begin{definition}
        Let $(X, \mathcal{A}, \mu)$ be a measure space. Then,
        \[\mathcal{L}^p(X, \mu) = \mathcal{L}^p(X) = \left\{f \colon X \to \mathbb{R} \mid f \text{ measurable and } \int_X |f|^p \ d\mu < \infty \right\},\]
        where $p \in [1, \infty)$. If $f \in \mathcal{L}^p(X)$, we define
        \[\lVert f \rVert_p = \left(\int_X |f|^p \ d\mu\right)^{1/p}.\]
    \end{definition}

    \begin{proposition}
        Let $(X, \mathcal{A}, \mu)$ be a measure space and $f, g \in \mathcal{L}^1(X)$. Then,
        \[\lVert f + g \rVert_1 \leq \lVert f \rVert_1 + \lVert g \rVert_1,\]
        and
        \[\lVert f + g \rVert_2 \leq \lVert f \rVert_2 + \lVert g \rVert_2.\]
    \end{proposition}
    \begin{proof}
        
    \end{proof}

    \begin{proposition}
        Let $(X, \mathcal{A}, \mu)$ be a measure space. Define the relation $\sim$ on functions $X \to \mathbb{R}$ by
        \[f \sim g \iff \mu (\{x \in X \mid f(x) \neq g(x)\}) = 0.\]
        Then, $\sim$ is an equivalence relation.
    \end{proposition}

    \begin{theorem}
        Let $(X, \mathcal{A}, \mu)$ be a measure space. Then, the vector space $\mathcal{L}^p(X)$ is complete.
    \end{theorem}
\end{document}